\chapter{Estrutura}
\label{cap:Estrutura}




\section{Módulos no espaço do núcleo}

A estratégia foi separar os diferentes componentes e a criação de um módulo de
\textit{debug} para que pudesse existir uma forma de acesso à monitorização por
parte do nível de utilizador, sem que exista a necessidade de criar ou alterar
chamadas ao sistema ou suas opções.

\paragraph{Estruturas}

A não criação de novas estruturas de dados, foi devido à existência das
estruturas de dados existentes no código do núcleo do sistema de operação
linux, que já foram bastante analisadas e existem com o propósito de serem
utilizadas por outros programadores.

\paragraph{Divisão do módulo} 

O módulo subdivide-se em 4 partes, a monitorização, o ``reservatório'' de
informação, o \textit{hook} da ligação com o \textit{bpf} e
comunicação/\textit{debug}.

\paragraph{Interacção com o LSF}
O sistema interage também com o \textit{LSF} pois existe uma conjunção
entre o filtro definido pelo utilizador e a captura do tráfego da aplicação a
ser monitorizada.


