\chapter{Conclusão}
\label{cap:conclusao}

O \textit{MRoP} é um módulo do núcleo que permite monitorizar as interacções de rede de um processo, sem que exista um conhecimento prévio dos portos que a aplicação utiliza.
Esta monitorização não tem quaisquer consequências para a aplicação, pois esta nem se desconhece que está a ser monitorizada, possibilitando ao administrador monitorizar aplicações sem necessidade de ter acesso ao código fonte.

Se analisarmos as aplicações do ponto de vista de segurança na rede é possível verificar se determinada aplicação está ou não a enviar indevidamente informação para o exterior.

Esta situação não é possível de observar com o recurso ao normal funcionamento da \textit{LibPCap} sem que se assista a uma monitorização da aplicação de forma intrusiva, o que pode originar situações de comportamentos erráticos e afectar negativamente o desempenho da aplicação e do sistema.
O recurso a esta nova forma de análise, ao fazer uso desta funcionalidade não intrusiva para a captura e análise do tráfego de um processo, representa um avanço relativamente à monitorização de rede efectuada através da \textit{LibPCap}.

O \textit{MRoP} ao oferecer a possibilidade de capturar exclusivamente os pacotes de um determinado processo, facilita as análises que se pretendem efectuar e reduz a sobrecarga neste tipo de sistemas, dispensando a \textit{LibPCap} de capturar o tráfego não pretendido, e de conhecer os protocolos e portos utilizados pela aplicação.


Esta funcionalidade é transparente para todas as ferramentas desenvolvidas com base no \textit{PCap}, pelo que todas podem dela usufruir.

Relativamente à sobrecarga introduzida, tendo como referência a monitorização já existente, esta revelou-se insignificante, melhorando substancialmente nos casos em que a incidência sobre sobre o tráfego de um único processo, reduz o trabalho realizado pelo \textit{LSF} e pela \textit{LibPCap}.

As vantagens do sistema criado assumem maior notoriedade a máquina se encontra ante uma carga mais elevada de trabalho ou um grande volume de tráfego de rede, dado manter o uso dos recursos na proporção aproximada apenas do tráfego do processo alvo.


\section{Trabalho Futuro}

Em termos de implementação existe a possiblidade de expandir e melhorar o suporte para multiplos processos a serem monitorizados.
Implementar suporte para controlo de concorrência utilizando \textit{spin locks}, ou outro método com suporte no núcleo de modo a não possibilitar a ocorrência de \textit{race conditions}.

Como trabalho futuro existe a possibilidade de integrar este trabalho no anterior trabalho \cite{duarte10,Farruca:2009}, formando uma ferramenta de monitorização distribuída com baixa sobrecarga.
 Como o sistema implementado permite apenas monitorizar protocolos assentes em \textit{TCP} e \textit{UDP}, uma contribuição seria suportar outros protocolos que se pretendam monitorizar (\textit{icmp, arp ,stp}, etc).
 Outra possibilidade será procurar optimizar a instrumentação, aplicando-a apenas a funções internas específicas dos protocolos monitorizados.
 Pretende-se brevemente a partilha destas alterações, submetendo este sistema para análise, pela comunidade utilizadora do sistema \textit{Linux} e possível implementação na versão principal do núcleo do \textit{Linux}.


