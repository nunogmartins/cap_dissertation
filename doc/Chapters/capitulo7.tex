\chapter{Conclusões e Trabalho Futuro}
\label{cap:conclusao}

Esta dissertação teve como objectivo a criação de uma extensão ao actual \textit{PCap}, utilizado no \textit{Linux}, de modo a restringir a captura dos pacotes referentes a uma determinada aplicação, contribuindo assim para a redução reduzindo da sobrecarga no sistema de monitorização e identificar os fluxos de rede de uma forma não intrusiva.

Para a realização da Monitorização de Rede orientada ao Processo ( \textit{MRoP} ), foi necessário para além de identificar os pontos e conhecer as razões que estiveram na génese do insucesso de anteriores trabalhos, criar alternativas que os permitissem ultrapassar.
Para atingir este desiderato foram estudados os mecanismos de monitorização internos ao núcleo e os sistemas de comunicação entre o núcleo e as aplicações em nível utilizador.

O estudo centrou-se nos mecanismos de monitorização ao nível do núcleo, pois estes permitem efectuar análises não intrusivas e devido à sua localização dispensam a permuta de dados entre o nível utilizador e o núcleo, o que a acontecer contribuiria para o aumento da sobrecarga.

Para executar a extensão ao \textit{LibPCap}, foi arquitectada uma solução utilizando o sistema de instrumentação dinâmica do núcleo, \textit{KProbes}, para a análise das interacções do processo alvo com o exterior.
Os dados relevantes desta interacção são adicionados num repositório de modo a que o sistema \textit{LSF} o pudesse consultar e decidir sobre quais os pacotes a capturar, com base nestas informações.

Esta extensão foi analisada funcionalmente através de programas que criavam, comunicavam e destruiam canais de comunicação, em que todas estas interacções eram registadas e verificadas.
Na avaliação funcional foram também executados programas efectuaram transferências utilizando os protocolos \textit{HTTP} e \textit{FTP}, enquanto decorriam outros fluxos na rede. 
O \textit{MRoP} foi aplicado a estes programas, de modo a capturar todo o trafego respeitante às transferências destes dois protocolos.
Os fluxos capturados foram analisados na ferramenta \textit{WireShark}, através do ficheiro que contém os fluxos de rede das transferências, verificando que apenas os respeitantes a estes dois protocolos existiam neste ficheiro.
Desde modo foi possível verificar a correcção dos protocolos e que apenas as interacções da aplicação com o exterior foram capturados.
Para além desta análise funcional, foi efectuada uma outra onde foram comparados os desempenhos do \textit{PCap}, com e sem esta nova extensão, a fim de determinar qual a sobrecarga introduzida pelo \textit{MRoP} na monitorização já existente.
Os resultados desta análise evidenciam que a sobrecarga é praticamente inexistente.

Nas secções seguintes são apresentados as principais conclusões da realização desta dissertação, assim como os possíveis melhoramentos e extensões ao \textit{MRoP}.


\section{Conclusões}
\label{sec:conclusoes}

O \textit{MRoP} é um módulo do núcleo que permite monitorizar as interacções de rede de um processo, sem que exista um conhecimento prévio dos portos que a aplicação utiliza.
Esta monitorização não tem quaisquer consequências para a aplicação, pois esta nem se desconhece que está a ser monitorizada, possibilitando ao administrador monitorizar aplicações sem necessidade de ter acesso ao código fonte.

Se analisarmos as aplicações do ponto de vista de segurança na rede é possível verificar se determinada aplicação está ou não a enviar indevidamente informação para o exterior.

Esta situação não é possível de observar com o recurso ao normal funcionamento da \textit{LibPCap} sem que se assista a uma monitorização da aplicação de forma intrusiva, o que pode originar situações de comportamentos erráticos e afectar negativamente o desempenho da aplicação e do sistema.
O recurso a esta nova forma de análise, ao fazer uso desta funcionalidade não intrusiva para a captura e análise do tráfego de um processo, representa um avanço relativamente à monitorização de rede efectuada através da \textit{LibPCap}.

O \textit{MRoP} ao oferecer a possibilidade de capturar exclusivamente os pacotes de um determinado processo, facilita as análises que se pretendem efectuar e reduz a sobrecarga neste tipo de sistemas, dispensando a \textit{LibPCap} de capturar o tráfego não pretendido, e de conhecer os protocolos e portos utilizados pela aplicação.

Esta funcionalidade é transparente para todas as ferramentas desenvolvidas com base no \textit{PCap}, pelo que todas podem dela usufruir.

Relativamente à sobrecarga introduzida, tendo como referência a monitorização já existente, esta revelou-se insignificante, melhorando substancialmente nos casos em que a incidência sobre sobre o tráfego de um único processo, reduz o trabalho realizado pelo \textit{LSF} e pela \textit{LibPCap}.

As vantagens do sistema criado assumem maior notoriedade, quando a máquina se encontra perante uma carga mais elevada de trabalho, ou um grande volume de tráfego de rede, dado manter o uso dos recursos na proporção aproximada apenas do tráfego do processo alvo.

\section{Trabalho Futuro}
\label{sec:future_work}
Como trabalho futuro existe a possiblidade de expandir e melhorar o suporte para multiplos processos a serem monitorizados.
Acresce ainda a possibilidade de controlo de concorrência, utilizando \textit{spin locks}, ou outro método com suporte no núcleo de modo a impossibilitar a ocorrência de \textit{race conditions}.

Considerando que o sistema implementado se limita a monitorizar protocolos acentes em \textit{TCP} e \textit{UDP}, poderia ver a sua contribuição alargada se abrange-se outros protocolos como \textit{icmp, arp, stp}, etc.

Outra possibilidade será procurar optimizar a instrumentação, restringindo a sua aplicação a funções internas especificas dos protocolos monitorizados.

Pretende-se partilhar o uso destas funcionalidades submetendo este sistema a analise da comunidade utilizador do sistema \textit{Linux} com vista à sua implementação na versão principal do núcleo do \textit{Linux}.
Considera-se a integração deste trabalho com o anterior \cite{duarte10,Farruca:2009}, com vista à obtenção de uma ferramenta de monitorização distribuída com baixa sobrecarga.
