\chapter{Conclusões e Trabalho Futuro}
\label{cap:conclusao}

Esta dissertação conseguiu o objectivo da criação de uma extensão ao actual \textit{PCap}/\textit{LSF}, utilizado no \textit{Linux}, de modo a restringir a captura dos pacotes referentes a uma determinada aplicação, contribuindo deste modo para a redução da sobrecarga no sistema de monitorização com uma nova funcionalidade e identificar os fluxos de rede de uma forma não intrusiva.

Para a realização da \textit{Monitorização de Rede orientada ao Processo} (\textit{MRoP}), foi necessário, para além de identificar os pontos e conhecer as razões que estiveram na génese do insucesso de anteriores trabalhos, criar alternativas que permitissem ultrapassá-los.
Para atingir tal objectivo, foram estudados os mecanismos de monitorização internos ao núcleo e os sistemas de comunicação entre o núcleo e as aplicações em nível utilizador.

O estudo centrou-se, principalmente, nos mecanismos de monitorização ao nível do núcleo, pois estes permitem efectuar análises não intrusivas e, devido à sua localização, dispensam a permuta de dados entre o nível utilizador e o núcleo, o que a não acontecer contribuiria para o aumento da sobrecarga.

Para executar a extensão ao \textit{PCap}, foi arquitectada uma solução utilizando o sistema de instrumentação dinâmica do núcleo (\textit{KProbes}), para a análise das interacções do processo alvo com o exterior.
Os dados relevantes desta interacção, são adicionados a um repositório, de modo a que o sistema \textit{LSF} os possa consultar e decidir quais os pacotes a capturar, com base nestas informações.

Esta extensão foi analisada funcionalmente através de programas que criavam, comunicavam e destruiam canais de comunicação, onde todas estas interacções eram registadas e verificadas.
Na avaliação funcional, foram igualmente executados programas que efectuavam transferências utilizando os protocolos \textit{HTTP} e \textit{FTP}, enquanto decorriam outros fluxos na rede. 
O \textit{MRoP} foi aplicado a estes programas, de modo a capturar todo o trafego respeitante às transferências destes dois protocolos, garantindo a compatibilidade e a possibilidade de continuar a utilizar as ferramentas existentes como o \textit{tcpdump} e o \textit{Wireshark}.
Foi possível constatar a correcção dos protocolos e verificar que apenas as interacções da aplicação alvo com o exterior, foram capturadas.
Para além desta análise funcional, foi efectuada uma outra, onde se compararam os desempenhos do \textit{PCap}, com e sem esta nova extensão, a fim de determinar qual a sobrecarga introduzida pelo \textit{MRoP} na monitorização já existente.
Os resultados apurados evidenciam que a sobrecarga é praticamente inexistente nos piores casos, trazendo vantagens na monitorização quando estivermos perante vários fluxos de dados irrelevantes para a análise.

Nas secções seguintes são apresentados as principais conclusões da realização desta dissertação, assim como as possíveis evoluções e extensões ao \textit{MRoP}.

\section{Conclusões}
\label{sec:conclusoes}

O objectivo proposto de criar uma extensão ao actual sistemas de monitorização genérico de rede, que inclua a monitorização das interacções de rede de um processo com base no identificador deste e efectuar a captura dos pacotes da aplicação, foi atingido, constatando-se ainda, que a sobrecarga imposta sobre o actual sistema de monitorização é praticamente irrelevante.

O \textit{MRoP} é um módulo do núcleo que permite monitorizar as interacções de rede de um processo, sem que exista um conhecimento prévio dos portos utilizados pela aplicação.
Esta monitorização é inócua para a aplicação, porquanto esta desconhece que está a ser monitorizada, possibilitando ao administrador monitorizar aplicações sem acesso ao código fonte.

Se analisarmos a aplicação, do ponto de vista de segurança na rede, é possível verificar se está, ou não, a enviar indevidamente informação para o exterior.
Esta situação, não é possível de observar com o recurso ao normal funcionamento da biblioteca \textit{PCap}, sem um conhecimento detalhado do seu funcionamente e protocolos, ou a uma monitorização da aplicação de forma intrusiva, o que pode originar comportamento errático e afectar negativamente o desempenho da aplicação e do sistema.
O recurso à introdução desta funcionalidade não intrusiva, para a captura e análise do tráfego de um processo, constitui um avanço relativamente à monitorização de rede efectuada através da biblioteca \textit{PCap}.

O \textit{MRoP}, ao oferecer a possibilidade de capturar exclusivamente os pacotes de um determinado processo ou de uma família de processos, facilita as análises a efectuar e reduz a sobrecarga neste tipo de sistemas, dispensando a biblioteca \textit{PCap} de capturar o tráfego não pretendido e, de conhecer os protocolos e portos utilizados pela aplicação.

Esta funcionalidade, é transparente para todas as ferramentas desenvolvidas com base no \textit{PCap}, pelo que todas podem dela usufruir.

Relativamente à sobrecarga introduzida, tendo como referência a monitorização de rede já existente, esta revelou-se insignificante mesmo nos piores casos, melhorando substancialmente aqueles em que incide sobre o tráfego de um único processo, reduzindo igualmente o trabalho realizado pelo \textit{LSF} e pela biblioteca \textit{PCap}.

As vantagens do sistema criado assumem maior notoriedade, quando a máquina se encontra perante uma carga mais elevada de trabalho, ou um grande volume de tráfego de rede e/ou muitos fluxos irrelevantes, dado manter o uso dos recursos na proporção aproximada apenas do tráfego do processo alvo.

\section{Trabalho Futuro}
\label{sec:future_work}
Como trabalho futuro existe a possiblidade de expandir e melhorar o suporte para multiplos processos a serem monitorizados.
Acresce ainda a possibilidade de verificar problemas de concorrência na presença de multiplos \textit{cores}/\textit{cpus}, e garantir a impossibilidade de ocorrência de \textit{race conditions}.

Considerando que o sistema implementado se limita a monitorizar protocolos acentes em \textit{TCP} e \textit{UDP}, poderia ver a sua contribuição alargada se abrangesse outros protocolos como \textit{icmp, arp, stp}, etc.

Outra possibilidade será procurar optimizar a instrumentação, evitando a instrumentação das funções \textit{sendto} e \textit{recvfrom}, restringindo a sua aplicação a funções internas específicas dos protocolos monitorizados.

Pretende-se partilhar o uso destas funcionalidades submetendo este sistema a análise da comunidade utilizadora do sistema \textit{Linux} com vista à sua implementação na versão principal do núcleo do \textit{Linux}.
Considera-se ainda a integração deste trabalho com o anterior \cite{duarte10,Farruca:2009}, com vista à obtenção de uma ferramenta de monitorização distribuída com baixa sobrecarga.
