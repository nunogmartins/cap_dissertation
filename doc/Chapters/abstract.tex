\abstract 
% The title ``Abstract'' font palatino. The text of the abstract should not exceed one page in 1.5 spacing as a reference.
Monitoring applications is a way to understand and analyze their behavior.
Is possible to monitor only some of the resources used by the application, such as cpu and memory utilization, IO devices (for example network interfaces), etc.

For a tool to monitor an applications network interactions uses a library, LibPcap, this library is specific to network capture with kernel support.
% For monitoring the applications interactions with the network is usual to link the program with a specific library, LibPcap. This library is specific to monitor the network. 
To capture only the data that we want, most of the time is necessary to use filters, this way we can separate the data that is relevant from that is not.

The LibPcap library provides a specific environment to make network monitoring, but for monitoring in a generic way the traffic of an application this one has serious performance issues. This issues are mainly because of the applications dynamic and to the communication structure between the applications and the external devices.

This dissertation aims to give a new filtering system to LibPcap based on the pid (process id) or the applications name, to capture only the packets that were sent or received by the monitored application.
To minimize the performance and overhead associated to the activity, some part of the development will be made available inside the Linux kernel. 

% Palavras-chave do resumo em Inglês
\begin{keywords}
Monitoring, network, application, LibPcap, packet capture, kernel instrumentation, filter, filtering
\end{keywords} 
