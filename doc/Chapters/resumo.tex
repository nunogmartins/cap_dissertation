\resumo 

A monitorização de aplicações permite analisar e compreender o comportamento destas. 
É possível monitorizar apenas parte dos recursos computacionais das aplicações tais como: utilização do cpu, da memória, dos dispositivos de \textit{IO} (em que nestes se enquadra a utilização dos dispositivos de rede), etc. 

As aplicações de monitorização de rede em geral utilizam a biblioteca \textit{LibPCap}, para capturar os fluxos de rede das interfaces de rede.
Como esta biblioteca é genérica de modo a abstrair o modo como o sistema de operação lida com a captura dos fluxos, por isso o núcleo do sistema de operação \textit{Linux} fornece o suporte à monitorização genérica de rede.

De modo geral o volume de tráfego obtido pela monitorização é elevado por isso é necessário filtrá-lo, para que apenas os fluxos relevantes sejam transmitidos para o monitor em nível utilizador.
Os filtros utilizados na monitorização são baseados no \textit{BPF}, que não tem suporte para a captura baseada nas interacções dos processos.
Se tivesse possivelmente muitos dos actuais filtros poderiam ser simplificados e o volume de dados transferidos para o monitor seria mais reduzido, contendo apenas os fluxos realmente relevantes melhorando o desempenho da monitorização.
Existem ferramentas que capturam monitorizam as interacções das aplicações com a rede, mas são demasiado específicas e oferecem fraco desempenho.

A solução proposta e implementada nesta dissertação resolve este problema através da Monitorização de Rede Orientada ao Processo (\textit{MRoP}), que através da extensão ao sistema de monitorização de rede do \textit{Linux} com um módulo no núcleo, instrumenta as chamadas ao sistema de rede, de modo a capturar as interacções das aplicações e com estes dados, filtrar os fluxos de rede da aplicação alvo.
Esta solução foi avaliada funcionalmente, de modo a verificar que apenas, e somente, os fluxos de dados pretendidos foram de facto capturados.
Para além desta avaliação, foi também efectuada a avaliação de desempenho que, no pior caso, aquele em que todos os pacotes da rede pertencem ao processo alvo, verificou-se que a sobrecarga foi praticamente nula.
No caso em que se pretendia um subconjunto dos pacotes, os pertencentes ao processo alvo, o desempenho da captura foi melhorado substancialmente, resultado já expectável.

%De forma a monitorizar a utilização de rede de uma aplicação é habitual utilizar-se a biblioteca \textit{LibPCap} especializada na monitorização de rede. Para capturar apenas os dados desejados em geral é necessário criar filtros de forma a separar os dados relevantes para a análise, daqueles que não o são. 

%A biblioteca \textit{LibPCap} proporciona um ambiente especifico para fazer esta monitorização, mas a forma de capturar genericamente o tráfego realizado por uma aplicação tem grandes problemas a nível de performance. Estes problemas são devido à dinâmica que as aplicações apresentam e à estrutura de comunicação entre as aplicações e os dispositivos de \textit{IO}.

%Esta dissertação propõe-se a criar um sistema de filtragem para a biblioteca \textit{libpcap} com base no \textit{pid} (\textit{Process id}) ou no nome da aplicação, de forma a obter apenas os pacotes da aplicação especificada. Com vista a mitigar os problemas anteriormente indicados, parte do desenvolvimento irá ser efectuado dentro do núcleo de sistema de operação.

% O texto do Resumo deverá ser escrito em fonte Palatino e não deve exceder uma página, usando um espaçamento e meio.  Note-se que a natureza do resumo depende do tipo de documento a produzir. 
% 
% O Resumo numa Preparação de Mestrado não pretende antecipar o resumo da dissertação após a sua elaboração. Deve, no entanto, permitir aferir que o aluno é capaz de resumir o problema a tratar e as principais contribuições previstas na sua dissertação, numa visão preliminar extraída do trabalho de preparação. 
% 
% O Resumo numa Tese de Mestrado deve conseguir sintetizar o problema, a solução proposta e os resultados da sua avaliação.
% 
% O aluno deve também apresentar este resumo em língua inglesa na página seguinte, como se indica. 

% Palavras-chave do resumo em Português
\begin{keywords}
Monitorização, rede, aplicação, libpcap, captura de pacotes, instrumentação do núcleo de sistema, filtro
\end{keywords}
% to add an extra black line
