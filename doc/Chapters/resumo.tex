\resumo 

A monitorização de aplicações permite analisar e compreender o seu comportamento, sendo possível monitorizar durante execuções reais os seus recursos computacionais, nomeadamente a utilização do \textit{cpu}, da memória, dos dispositivos de \textit{IO} (incluindo os dispositivos de rede), etc. 

As ferramentas de monitorização de rede, em geral, utilizam a biblioteca \textit{LibPCap}, de modo a capturar os fluxos das interfaces de rede.
Como esta biblioteca é genérica, permite abstrair o modo como o sistema de operação lida com a captura dos fluxos, sendo que no \textit{Linux}, o suporte é garantido pelo sistema de captura e filtragem de pacotes de rede \textit{Linux Socket Filter}.


De um modo geral, o volume de tráfego obtido pela monitorização é elevado, sendo necessário filtrá-lo, de forma a que apenas os fluxos de dados relevantes sejam transmitidos para o monitor, em nível utilizador.
Quer no \textit{libPCap} quer no suporte dos sistemas não existe suporte para a captura baseada nas interacções dos processos.
%Caso tivesse, muitos dos actuais filtros poderiam ser simplificados e o volume de dados transferidos para o monitor seria mais reduzido e relevante, melhorando o desempenho da monitorização.
%Existem ferramentas que monitorizam as interacções das aplicações com a rede, sendo demasiado específicas e de fraco desempenho.
Esta funcionalidade é bastante útil para os utilizadores e com vantagens na sobrecarga do sistema e desempenho.

A solução proposta pretende resolver esta problemática, através da Monitorização de Rede Orientada ao Processo (\textit{MRoP}), estendendo o sistema de monitorização de rede do \textit{Linux}, por meio da introdução de um módulo no núcleo, permitindo capturar as interacções das aplicações e filtrando os fluxos de rede da aplicação alvo.
Esta solução foi avaliada funcionalmente, verificando-se que apenas os fluxos de dados pretendidos existiam na captura.
Para além desta avaliação, foi igualmente realizada uma outra de desempenho, que dependeu do conjunto de pacotes capturados, relativos ao processo alvo.




%No caso em que se pretendia um subconjunto dos pacotes, os pertencentes ao processo alvo, o desempenho da captura foi melhorado substancialmente, resultado já expectável.

%De forma a monitorizar a utilização de rede de uma aplicação é habitual utilizar-se a biblioteca \textit{LibPCap} especializada na monitorização de rede. Para capturar apenas os dados desejados em geral é necessário criar filtros de forma a separar os dados relevantes para a análise, daqueles que não o são. 

%A biblioteca \textit{LibPCap} proporciona um ambiente especifico para fazer esta monitorização, mas a forma de capturar genericamente o tráfego realizado por uma aplicação tem grandes problemas a nível de performance. Estes problemas são devido à dinâmica que as aplicações apresentam e à estrutura de comunicação entre as aplicações e os dispositivos de \textit{IO}.

%Esta dissertação propõe-se a criar um sistema de filtragem para a biblioteca \textit{libpcap} com base no \textit{pid} (\textit{Process id}) ou no nome da aplicação, de forma a obter apenas os pacotes da aplicação especificada. Com vista a mitigar os problemas anteriormente indicados, parte do desenvolvimento irá ser efectuado dentro do núcleo de sistema de operação.

% O texto do Resumo deverá ser escrito em fonte Palatino e não deve exceder uma página, usando um espaçamento e meio.  Note-se que a natureza do resumo depende do tipo de documento a produzir. 
% 
% O Resumo numa Preparação de Mestrado não pretende antecipar o resumo da dissertação após a sua elaboração. Deve, no entanto, permitir aferir que o aluno é capaz de resumir o problema a tratar e as principais contribuições previstas na sua dissertação, numa visão preliminar extraída do trabalho de preparação. 
% 
% O Resumo numa Tese de Mestrado deve conseguir sintetizar o problema, a solução proposta e os resultados da sua avaliação.
% 
% O aluno deve também apresentar este resumo em língua inglesa na página seguinte, como se indica. 

% Palavras-chave do resumo em Português
\begin{keywords}
Monitorização, rede, aplicação, LibPCap, captura de pacotes, instrumentação do núcleo de sistema, filtro
\end{keywords}
% to add an extra black line
