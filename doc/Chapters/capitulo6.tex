\chapter{Avaliação}
\label{cap:avaliacao}

Antes de se utilizar um novo sistema, é conveniente proceder-se a testes alargados de verificação, para que se tome conhecimento do seu correcto funcionamento e das sobrecargas introduzidas, para que deste modo possa ser utilizado como uma mais–valia.
O mecanismo implementado (\textit{MRoP}) foi avaliado funcionalmente, através de diversos testes relativamente aos dados de entrada nos diversos componentes que o constituem.
Como existem diversas fontes de dados para o \textit{MRoP} (dados de controlo, dados do processo alvo de monitorização, fluxos de rede, etc.), foram criados vários testes de modo a que todas estas fontes de dados sejam analisadas.
Para além destas fontes directas de dados foi, também, verificada a correcção de que todos os fluxos de dados obtidos, através da monitorização de rede, são exclusivos ao processo alvo.
Assim todos estes testes e avaliações foram executados com sucesso e serão apresentados na secção \ref{sec:eval_functional}.

Como um dos principais objectivos desta dissertação é criar um mecanismo com melhor desempenho que os anteriores, apresentados na secção \ref{sect:outras_abordagens}, foi assim necessário verificar se este tinha sido atingido.
Desta forma foi necessário efectuar testes de desempenho sobre \textit{MRoP}, de modo a verificar se este objectivo tinha sido atingido.
Os testes de desempenho serão apresentados na secção \ref{sec:eval_performance}, incidindo sobre o desempenho global na transferência de 1GB de dados, através de protocolos conhecidos, bem como através de testes de desempenho que visam o mecanismo de instrumentação e a estrutura de dados escolhida para manter o estado do processo alvo.

Por último, serão apresentadas algumas conclusões sobre as análises efectuadas ao \textit{MRoP} na secção \ref{sec:five_chap_conclusion}.

\section{Avaliação Funcional}
\label{sec:eval_functional}

A análise funcional ao \textit{MRoP} teve várias vertentes, uma vez que este mecanismo recebe dados de diferentes fontes.
Assim os testes visaram a verificação dos dados indicados pelo administrador, pelo processo alvo e os das estruturas do núcleo.
Foi necessário garantir que não existiam falhas nos mecanismo de entrada de dados ao \textit{MRoP}, porque como este está a executar no núcleo tem acesso a todo o sistema, assim qualquer problema na validação de dados poderá comprometer o sistema tanto a nível de segurança como a nível de disponibilidade.



\subsection{Teste ao componente de controlo}

Como o \textit{MRoP} foi desenvolvido para ser um módulo para o núcleo, é necessário que seja adicionado a este através do programa \textit{insmod} ou semelhante.
Após a adição do \textit{MRoP} ao núcleo, para funcionar é necessário ser um utilizador com permissões de administrador (\textit{root}), e escrever em determinados ficheiros criados no \textit{DebugFs} para o controlo do \textit{MRoP}.
Dado que o \textit{MRoP} é parte do núcleo, é necessário garantir que qualquer dado que lhe é passado não compromete a segurança nem a disponibildade do sistema, por isso as funções que recebem dados oriundos do sistema de controlo efectuam verificações antes de os passar para as restantes componentes do \textit{MRoP}.
Como os dados de controlo são cadeias de caracteres, a função de verificação limita o tamanho máximo de caracteres, baseado no valor máximo expectável para a utilização do ficheiro.
Assim cadeias de caracteres maiores que um determinado tamanho são truncadas, não existindo a possibilidade de explorar falhas de \textit{overflow}.
O valor zero (\textit{0}) serviu como valor inicializador, sendo que valores superiores a este, indicadores de que existe uma opção activa no \textit{MRoP}.
Para além destas análise, foram também verificadas que as permissões dos ficheiros estão de acordo com o estritamente necessário, e que apenas estão disponíveis ao utilizador \textit{root}.

\subsection{Obtenção do estado dos canais do processo alvo}

A análise funcional de detecção dos protocolos, portos e endereços locais, foi efectuada recorrendo à criação de dois conjuntos de programas \textit{cliente/servidor}.
Para o primeiro conjunto foi criado um servidor e um cliente, ambos utilizando o protocolo \textit{tcp}, em que o programa servidor esperava conexões numa porta e endereço pré-definidos.
Como o \textit{MRoP} instrumenta as chamadas ao sistema \textit{connect}, \textit{accept}, \textit{bind}, \textit{recvfrom}, \textit{sendto} e a função \textit{sock\_close}, é possível capturar todas as operações relativamente às comunicações de rede utilizando os protocolos \textit{tcp} e \textit{udp}.
Nas chamadas ao sistema \textit{bind} e \textit{connect}, um dos seus argumentos é uma estrutura \textit{sockaddr} que contém informações sobre o endereço e porto, sendo que na função \textit{connect} estes são relativos ao destino, e na função \textit{bind} são relativos à origem.
Num \textit{socket} recentemente criado, as estruturas \textit{socket} e \textit{sock} que lhe pertencem, ainda não contêm os dados relativos aos portos e endereços.
Assim os dados anteriormente mencionados, são de extrema utilidade para a construção do repositório do \textit{MRoP}, de modo a capturar apenas o tráfego respeitante ao processo alvo.
Como os dados são provenientes do processo alvo, não é possível confiar que estão correctos, assim caso a função a executar retorne com um valor de erro, os valores adicionados são retirados do repositório, ficando novamente num estado consistente com a realidade.
De modo a verificar que os dados que se encontram no repositório estão completos e correctos, é efectuada a leitura de todos os valores do repositório e apresentados através da função \textit{printk} para o terminal do núcleo, ou então são apresentados no ficheiro criado para o efeito no sistema de ficheiros virtual \textit{DebugFs}, verificando se são os mesmo que a aplicação \textit{netstat} declara para o processo alvo.

Quando o processo alvo já se encontra em execução e se inicia a monitorização, caso os \textit{sockets} estejam em utilização contêm os dados relativos aos protocolos, portos e endereços dos \textit{sockets} e, por isso consideram-se correctos e válidos, sendo automaticamente adicionados ao repositório.



%\subsection{Avaliação das estruturas dos \textit{sockets} no núcleo}
%
%Os testes no núcleo utilizaram a função \textit{printk} para mostrar as informações necessárias no registo do núcleo, que pode ser acedido em nível utilizador através do programa \textit{dmesg} ou directamente do ficheiro \textit{messages} no directório $\backslash$\textit{var}$\backslash$\textit{log}.
%
%Criaram-se dois sistema simples de cliente/servidor, um utilizando canais \textit{tcp} e o outro canais \textit{udp}, de modo a verificar como os diferentes \textit{sockets} apresentam os dados referentes aos portos, endereços e protocolos no núcleo.
%Os servidores esperavam conecções numa porta escolhida previamente, permitindo que os clientes efectuassem conecções para estes.
%Nos servidores e nos clientes eram apresentados os endereços de memória da estrutura de dados (\textit{struct sockaddr}), que contém as propriedades dos \textit{sockets}, tais como, porto, endereço, protocolo, etc.
%
%A utilização de \textit{KRetProbes} permitiu que se garantisse a correcção dos dados, uma vez que se os dados que não estivessem de acordo com o esperado, o núcleo através do valor de retorno indicaria que existia algum problema.
%Estas situações estão contempladas dado que na função de \textit{handler} de retorno, um dos primeiros dados a verificar é o valor de retorno indicado pela função instrumentada, caso este valor seja de erro, todas as alterações previamente introduzidas no repositório serão removidas, deixando assim o repositório consistente com o núcleo.

\subsection{Avaliação de monitorização de rede}

Foi efectuada uma verificação da correcção dos dados transferidos entre dois processos, um cliente e outro servidor.
Estes testes foram apenas efectuados para protocolos assentes sobre \textit{tcp}, pois nestes existe a certeza que, devido ao sistema de controlo de comunicação não existem falhas, o que permite analisar a transferência desde o seu início até ao seu término.

A execução do teste consistiu na transferência de um ficheiro entre duas máquinas, ligadas por um \textit{switch} com portas a 100 \textit{Mbit/s}, através do protocolo aplicacional \textit{ftp}, enquanto existiam outros fluxos de rede de outros processos.
Foi escolhido este protocolo pois, apesar de ser necessário conhecer \textit{à priori} a porta de comunicação com o servidor, a transmissão de dados é efectuada noutra porta negociada dinamicamente, permitindo demonstrar também o potêncial do \textit{MRoP}.
Todas as comunicações remotas referentes a esta transferência foram monitorizadas através do programa \textit{tcpdump}, com o módulo do núcleo \textit{MRoP} activo e com a indicação do processo a monitorizar.
Esta monitorização foi guardada através do \textit{tcpdump} num ficheiro para posterior análise através do \textit{Wireshark}.
Utilizando o ficheiro capturado no \textit{Wireshark}, foi possível observar que apenas os fluxos de rede da transferência existiam no ficheiro.
Após esta verificação, foi efectuada a recuperação do ficheiro transmitido através da agregação dos dados presentes nos pacotes, exceptuando os cabeçalhos.
Esta recuperação foi guardada num ficheiro temporário, de modo a poderem ser aplicadas funções de sintese (\textit{md5} e \textit{sha1}), com o objectivo de verificar que o conteúdo dos pacotes recuperado era exactamente igual ao ficheiro original, e à sua transferência.
Esta verificação foi confirmada através do mesmo valor retornado para os três ficheiros, para cada uma das funções de síntese utilizadas.

Foi também necessário verificar a correcção de transferências sobre o protocolo \textit{udp}, por isso realizou-se um teste utilizado o programa \textit{iperf} com recurso a \textit{sockets} \textit{udp}.
Para a realização deste teste foi utilizada a mesma infra-estrutura de rede, em que numa das máquinas foi executado o \textit{iperf} em modo servidor utilizando \textit{sockets udp}, e na outra máquina foi executado o modo cliente utilizando \textit{sockets} \textit{udp}.
Foi efectuada a monitorização de rede na máquina que executou o \textit{iperf} em modo cliente, através do programa \textit{tcpdump} com o módulo do núcleo activo e com a indicação do identificador do processo \textit{iperf}, enquanto existiam outros fluxos de rede de outros processos.
O resultado da monitorização pelo \textit{tcpdump} foi guardado num ficheiro, que indicou que o total de \textit{bytes} correctamente recebidos pelo \textit{iperf} em modo servidor, foram também obtidos pela monitorização.
Neste ficheiro não existiam mais fluxos de dados que não os respeitantes à utilização do \textit{iperf}.

Como anteriormente referido, aquando dos testes existiram outros fluxos de rede, nomeadamente de tráfego \textit{web}, de aplicações de conversação instantânea, entre outros, de modo a comprovar a eficácia do \textit{MRoP}.

\section{Avaliação do desempenho}
\label{sec:eval_performance}

Tendo presente a avaliação do desempenho, foram efectuados diversos testes com o objectivo de avaliar a sobrecarga gerada pela introdução do \textit{MRoP}.
Estes testes basearam-se na recepção ou transmissão de \textit{1 GigaByte} de dados, utilizando diferentes programas e protocolos, entre duas máquinas ligadas directamente.
Ambas as máquinas, que se optou por designar de máquina 1 e máquina 2, procederam à transmissão/recepção de dados, utilizando cada uma, apenas, um processador activo de 2 e de 2.6 Ghz, respectivamente.
As máquinas anteriormente descritas encontravam-se conectadas directamente, por interfaces de rede a 100 MBit/s, ficando uma das máquinas responsável pela execução dos servidores \textit{ftp}, \textit{http} e \textit{iperf}, e a outra pelos respectivos clientes.
A versão do sistema de operação utilizado, em ambas as máquinas, correspondeu ao 2.6.39, sendo que na máquina 1 foram introduzidas as modificações, para incluir o \textit{hook} do \textit{MRoP} e as suas funções auxiliares, enquanto na máquina 2 se executou o sistema original.

\subsection{Desempenho do \textit{MRoP}}


Na execução destes testes, foram efectuadas dez iterações, isto é, cada teste foi executado dez vezes, para cada experiência considerada, de modo a obter um valor médio e um desvio padrão considerado aceitável.
Os testes efectuados, em particular os primeiros, ilustram situações em que não há grande vantagem em ter o sistema o \textit{MRoP} activo, com vista a medir a sobrecarga do \textit{MRoP}.
Os resultados obtidos constam nas tabelas \ref{tab:desempenho} e \ref{tab:overhead}.

Os primeiros quatro testes foram efectuados utilizando apenas uma conexão ao servidor, enquanto o 5º e o 6º testes utilizaram mais uma comunicação, de modo a aumentar o peso sobre o processador e o número de pacotes a circular entre as máquinas.
Desta forma, foi possível identificar a sobrecarga exercida enquanto o \textit{tcpdump} executava e capturava todos os pacotes ou apenas um subconjunto destes, ou seja, os pacotes relativos aos processos alvo no novo sistema.

Na tabela \ref{tab:desempenho}, a coluna "Original" corresponde aos valores resultantes dos tempos médios das execuções das transferências na ausência de monitorização.
Na mesma tabela e na coluna "Com \textit{TcpDump}", é apresenta a média dos tempos de transferência com a captura total do tráfego utilizando a \textit{LibPCap}/\textit{LSF} original, enquanto que a coluna identificada com "Com \textit{TcpDump} e \textit{MRoP}" regista a média dos tempos para a transferência com captura pelo tcpdump e o módulo \textit{MRoP} desenvolvido no núcleo, de forma a capturar, apenas, o tráfego da transferência do processo alvo.
Nos primeiros quatro testes é possível verificar que a utilização do \textit{MRoP}, aumentou de forma insignificativa o tempo de execução (figura \ref{fig:tests_graphics}).

\begin{figure}[!htbp]
\centering
\includegraphics[scale=0.6]{testes.jpg}
\caption{Testes de desempenho efectuados ao MRoP}
\label{fig:tests_graphics}
\end{figure}

É igualmente possível observar que no 1º e 3º testes, aquando da utilização do \textit{tcpdump}, a execução sem o \textit{MRoP}, mostrou-se ligeiramente mais rápida, como se pode verificar na tabela \ref{tab:desempenho} e \ref{tab:overhead}.
\begin{table}[!htb]
\begin{center}
\caption{Tempos médios em segundos (s)}
\begin{tabular}{ | c | c | c | c |  }
\hline
Teste & \hspace {0.3cm} Original \hspace {0.3cm}& \hspace {0.2cm} Com TcpDump \hspace {0.2cm} & Com TcpDump e MRoP \\
\hline
1GB - FTP$^{1}$ & 91.8508	& 91.8500 & 91.8854 \\
1GB - HTTP$^{2}$ & 91.6391 & 91.6472 & 91.6674 \\ 
IPerf - 1GB TCP$^{3}$ & 91.3790	& 91.2535	& 91.2672 \\
IPerf - 1GB UDP$^{4}$ & 89.7975 & 89.8007 & 89.8464 \\
\hline
\hline
1GB HTTP - 2 conexões$^{5}$ & 182.1573 & 188.7156 & 182.0161 \\
IPerf - 1GB UDP 2 conexões$^{6}$ & 179.4930 & 179.6280 & 179.6369 \\
\hline
\end{tabular}
\label{tab:desempenho}
\end{center}
\end{table}
Esta situação deve-se ao facto de, quando a máquina se encontra em sobrecarga, leva ao aumento do tamanho médio dos pacotes, reduzindo o seu número e o volume de dados transferidos, em virtude da diminuição dos seus cabeçalhos.
Este comportamento já tinha sido detectado numa dissertação anterior~\cite{Farruca:2009}.

Nos 5º e 6º testes, como o tráfego na interface é duplicado e o \textit{tcpdump} tem que capturar todos os pacotes, é possível evidenciar a sobrecarga exercida por estas cópias de dados e consequentes transferências (para nível utilizador) face ao novo sistema onde apenas captura um fluxo de dados.



Na tabela \ref{tab:overhead} e na figura \ref{fig:tests_overhead} é possível observar que, para o teste 5, a sobrecarga do \textit{tcpdump} atinge os 3.6\% face ao original, enquanto que a sobrecarga do \textit{tcpdump} com o \textit{MRoP}, permitiu uma ligeira melhoria face ao original (-0.0775\%).
Conclui-se, portanto, que quando o fluxo de dados que não pretendemos capturar aumenta consideravelmente, torna-se mais vantajoso utilizar o \textit{MRoP}, do que capturar todos os pacotes, na medida em que se minimiza a sobrecarga, capturando apenas os dados relevantes, evitando-se a identificação e filtragem dos pacotes pertencentes ao processo alvo em nível utilizador, tendo como consequência uma sobrecarga adicional.

\begin{figure}[!ht]
\centering
\includegraphics[scale=0.7]{overhead.jpg}
\caption{Sobrecarga nos testes 5 e 6 }
\label{fig:tests_overhead}
\end{figure}


\begin{table}[!htb]
\begin{center}
\caption{Sobrecarga das transferências (valores em percentagem)}
\begin{tabular}{ | c | c | c |}
\hline
Teste & \hspace {0.3cm} TcpDump \hspace {0.3cm} & TcpDump com MRoP  \\

\hline
1GB - FTP$^{1}$ & -0.0009  & 0.0377  \\
1GB - HTTP$^{2}$ & 0.0088 &  0.0309   \\
IPerf - 1GB TCP$^{3}$ & -0.1373 &  -0.1223   \\
IPerf - 1GB UDP$^{4}$ & 0.0036 & 0.0545 \\
\hline
\hline
1GB HTTP - 2 conexões$^{5}$ & 3.6003 & -0.0775   \\
IPerf - 1GB UDP 2 conexões$^{6}$ & 0.0752 & 0.0802   \\
\hline
\end{tabular}
\label{tab:overhead}
\end{center}
\end{table}

Os resultados da sobrecarga do \textit{tcpdump} com o \textit{MRoP}, em todos os casos analisados, foram sempre inferiores a 0.15\%, chegando mesmo a ser de 0.0309\%.
Se estes resultados forem comparados com os obtidos em \cite{Farruca:2009}, pode verificar-se que são substancialmente melhores.
Efectuando a comparação com o resultado obtido no 2º teste, da transferência por \textit{HTTP}, não por ser o melhor resultado obtido, mas por o teste ser identico, pode verificar-se a existência de uma melhoria no desempenho de 64 vezes.

Como foi anteriormente referido, os testes realizados apenas com o \textit{tcpdump} capturam todo o tráfego e, como não existia a monitorização do processo alvo, não era efectuada a filtragem dos pacotes.
Desta forma os valores apresentados para o \textit{tcpdump} são meramente indicativos relativamente ao \textit{tcpdump} com o \textit{MRoP} activo, uma vez que lhe falta a componente de filtragem dos dados, aumentando o tempo de execução e a sua sobrecarga.

\subsection{Desempenho da estrutura de dados}

Para além das avaliações anteriormente descritas, tornou-se essencial analisar o comportamento da estrutura de dados utilizado no componente “estado do processo”, de modo a verificar o seu desempenho.
Assim para esta análise, foi elaborado um teste para determinar o desempenho da estrutura de dados, em relação às inserções e remoções.
Este teste utiliza o sistema de alta resolução de temporizadores (\textit{HRTimer})\cite{hrtimerKernel}, presente no núcleo do sistema de operação.

Para decidir qual o número de elementos a ser utilizado para este teste, foi verificado qual o valor máximo de descritores de canais que um processo pode ter.
O valor foi obtido através da função \textit{getrlimits}, que indicou que o valor máximo de descritores de canais abertos, para um processo, é de 1024 canais.
Dado que este valor (1024) é o máximo de canais por processo, considerou-se um optimo valor para verificar o comportamento da estrutura de dados no pior caso, no cenário em que todos os descritores de canais são \textit{sockets} e estão activos.
Os valores obtidos através deste teste demonstrarão os valores máximo espectáveis, em que se assume que o processo alvo está a utilizar o número máximo de canais de rede.

Assim o teste consistiu em obter o tempo anterior e posterior à inserção dos 1024 elementos, representando outros tantos portos/endereços, afim de determinar o tempo decorrido.
De igual modo, foi calculado o tempo de remoção dos referidos elementos.
Os resultados obtidos estão reproduzidos na tabela \ref{tab:tree_info}.
 
\begin{table}[!htb]
\begin{center}
\caption{Custo das operações (tempos em nanosegundos)}
\begin{tabular}{ | r | c | c | }
\hline
\hspace{1cm} Teste \hspace{1.5cm} & \hspace{1cm}Duração\hspace{1cm} &  Média por
elemento \\
\hline
Adição de 1024 elementos & 869 244 & 848.8711 \\
\hline
Remoção de 1024 elementos & 675 086 & 659.2637\\
\hline

\hline
\end{tabular}
\label{tab:tree_info}
\end{center}
\end{table}

Pode verificar-se, pela tabela \ref{tab:tree_info}, que a inserção de um elemento na árvore é inferior a 1 microsegundo, demonstrando que a estrutura utilizada não introduz uma elevada sobrecarga.
A inserção de dados no repositório, não é efectuada com um desempenho constante, dado que quando é necessário efectuar um rebalanceamento da árvore, o processo de inserção é mais demorado devido à necessidade de efectuar rotações na árvore.
Para além de estabelecer um bom compromisso de desempenho e utilização de memória, a sua disponibilidade de utilização no núcleo do sistema, possibilitou ter um elevado grau de confiança na sua utilização.

O tempo médio despendido na procura do elemento com o menor valor de chave, nos 1024 elementos adicionados, foi de 1327 nanosegundos.
Com este valor é possível verificar que para efectuar 10 iterações de procura na árvore, incorre-se numa penalização de 1.3 microsegundos.
Verifica-se que o tempo médio de procura de elementos na estrutura, é sempre menor ou igual a 1.3 microsegundos.
Considerando que a maioria das aplicações não utiliza tantos portos em simultâneo, são expectáveis tempos inferiores em aplicações reais.

O teste de desempenho realizado demonstra que, mesmo nas piores condições, ou seja nas condições mais extremas, consegue-se obter um desempenho aceitável, tendo em conta que se utilizou uma estrutura de dados dinâmica, em que se tem flexibilidade para modificar os seus dados.


\subsection{Desempenho do Sistema de instrumentação}
Sendo a instrumentação das chamadas ao sistema um dos pontos fundamentais na execução do \textit{MRoP}, a análise ao seu comportamento é bastante importante, na medida em que é necessário verificar se a introdução deste tipo de instrumentação irá produzir uma elevada penalização sobre o sistema de operação.
A titulo exemplificativo da sobrecarga introduzida pelo \textit{KRetProbe} em cada função instrumentada, foi adicionado \textit{KRetProbe} à chamada ao sistema \textit{getpid}.
Esta chamada ao sistema foi escolhida, porque é uma função muito simples que apenas devolve o identificador do processo que a invocou.
Este teste consistiu em avaliar o tempo decorrido entre o início e o fim do total das chamadas, com e sem o \textit{KRetProbe}, de forma a avaliar a sobrecarga e verificar se coincide com o indicado pelos criadores do sistema.


\providecommand{\e}[1]{\ensuremath{\times 10^{#1}}}

\begin{table}[!htb]
\begin{center}
\caption{Duração das chamadas em segundos}
\begin{tabular}{ | c | c | c | c |}
\hline
Teste & Original & Com \textit{KRetProbe} & Sobrecarga por chamada\\
\hline
100 000 000 chamadas & 12.65 &  73.6600 & 610.10\e{-9}\\
1 000 000 000 chamadas & 126.85 & 737.2100 & 610.36\e{-9}\\
\hline
\end{tabular}
\label{tab:kprobes_info}
\end{center}
\end{table}

O valor de referência obtido pelos criadores do \textit{KProbes}, referentes ao \textit{KRetProbe} sem optimizações, é de 0.7 microsegundos\cite{KProbeKernel}, sendo que o valor médio obtido foi de 0.61 microsegundos, ou seja, ligeiramente inferior, visto que a máquina de referência apresenta uma frequência de \textit{cpu} inferior à máquina onde foram realizados estes testes.

Consideram-se estes valores bastante aceitáveis e espera-se que tenham ainda reduzido impacto no desempenho normal do sistema.
Note-se ainda que a instrumentação só é introduzida aquando do carregamento do \textit{MRoP}, de modo a executar a monitorização com esta nova funcionalidade.

\section{Conclusão}
\label{sec:five_chap_conclusion}

Com os testes efectuados ao nível funcional e ao nível de desempenho ao \textit{MRoP} verificou-se que este mecanismo oferece um desempenho aceitável nas condições mais adversas, o que indica que para as condições das aplicações reais o seu desempenho será superior, enquanto que a sobrecarga introduzida no sistema poderá ser inferior ao constatado.

Pode verificar-se em relação ao trabalho \cite{Farruca:2009}, que a sobrecarga gerada pelo \textit{MRoP} foi muito inferior, demonstrando que apesar da dificuldade de trabalhar no núcleo, criar um sistema de monitorização de rede orientado ao processo através de um mecanismo não intrusivo, que assim que possível rejeite a captura de pacotes, reduz consideravelmente a sobrecarga no sistema.

Existe a possibilidade de reduzir a sobrecarga no sistema se ao invés de se instrumentar as chamadas ao sistema, forem instrumentadas as funções especificas relacionadas com a pilha de protocolos \textit{tcp/ip}.

