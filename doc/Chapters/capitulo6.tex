\chapter{Avaliação}
\label{cap:avaliacao}

\section{Ferramenta de testes e}
%Como efectuar a monitorização

Foi desenvolvida uma aplicação para testar o desenvolvimento da ferramenta.
Esta aplicação 

\subsection{Test Unit}

De forma a testar o repositório de dados foram efectuados diferentes \textit{unit tests}, de forma a assegurar que todas as alterações efectuadas na ferramenta ficavam de correctas.

\subsection{Aplicação Monitora}
\label{sub:monitor_app}

Para poder mais facilmente efectuar os testes de avaliação, foi criado uma ferramenta em nível utilizador que permite lançar a aplicação e configurar automaticamente o sistema para a monitorizar. Esta verifica o identificador do processos e o momento em que se dá o inicio e o fim da sua execução, de forma a iniciar e terminar a monitorização quando necessário.

\subsection{Aplicação de testes de performance}

Para ajudar na avaliação da ferramenta desenvolvida foi criado um conjunto de aplicações independentes ( scripts e aplicações ) para monitorizar a actividade de algumas aplicações que se consideraram pertinentes no processo de avaliação do desempenho da ferramenta.

A aplicação desenvolvida \textit{manager} tem de ser executada sob o controlo do utilizador \textit{root}, devido à necessidade de executar processos que só este utilizador tem acesso. Estes processos anteriormente mencionados são \textit{insmod}, \textit{rmmod} e \textit{tcpdump}.

Os \textit{scripts} criandos e bash serviram para obter o número de dados e pacotes transferidos na interface, bem como fazer a separação dos tempos que as aplicações de testes executaram para conseguir automatizar o processo de execução e recolha de dados dos testes.


\section{Temporizadores}

Temporizadores no núcleo do sistema.


\subsection{Temporizadores de Alta-Resolução}

\textit{HrTimer}
