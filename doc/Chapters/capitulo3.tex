\chapter[Plano de trabalho]{Plano de trabalho} \label{cap:plano}



% Este capítulo deve apresentar um plano de trabalho que contenha uma visão de sequência das tarefas e passos a desenvolver pelo aluno bem como os seus objectivos para a elaboração da dissertação. Este plano deve ainda conter indicadores temporais das tarefas a desenvolver, bem como uma indicação das suas inter-dependências. 
% 
% As tarefas apresentadas neste plano podem ser livremente caracterizadas, devendo a sua leitura propiciar uma ideia clara sobre o seu enquadramento, como a seguir se sugere:
% 
% Tarefas que envolvem aprofundamento ou refinamento de pesquisa bibliográfica para enriquecimento da visão de trabalho relacionado ou de domínio de estado da arte;
% 
% Tarefas associadas a modelação ou formalização do problema;
% 
% Tarefas associadas a protótipos de implementação ou de prova de conceito;
% 
% Tarefas associadas a ensaios experimentais e tratamento de resultados obtidos ou formas previstas para avaliação das contribuições.
% 
% O plano deve permitir ao leitor uma visão clara sobre a relevância das tarefas previstas para os objectivos e contribuições esperadas para a elaboração da dissertação.

Durante o período de elaboração irão ser efectuadas 4 fases, desenho do protótipo, implementação, avaliação e elaboração do relatório. Após as três primeiras fases, a quarta será foca apenas na escrita do documento final. 
\begin{enumerate}

\item Para o desenho do protótipo irão ser estudadas as formas de incorporação da funcionalidade de captura do tráfego de uma aplicação na biblioteca \textit{LibPcap}, para que possa ser utilizado em conjunto com as restantes funcionalidades desta biblioteca. 

\item A implementação e sucessivas iterações da implementação irá durar sensivelmente dois meses. Este tempo serve para poder implementar no núcleo de sistema de operação do \textit{Linux} os módulos necessários à ferramenta. Estes módulos foram discutidos na abordagem proposta em \ref{sec:abordagem_proposta}. 

\item A avaliação da ferramenta irá incidir principalmente sobre a perturbação que esta cria no sistema, e comparações com anteriores abordagens. Para que todo o conhecimento obtido seja capturado no documento final, durante estas fases irá ser escrito algumas das partes mais relevantes do trabalho. 

\item Na última fase, a escrita, este tempo será completamente dedicado à escrita do documento final. 

\end{enumerate}

\begin{landscape}
\begin{figure}[h]
       \label{fig:schedule}
       \includegraphics[scale=1.1 
%  , angle=180
]
%       [height=2.5in]
	{gantt}
       \caption{Plano de trabalho}
\end{figure}
\end{landscape}

% Uma referência~\cite{MorrisEbert02}

% Uma referência~\cite{LuMorris03}
 
%  Uma referência~\cite{Haber-1990-VIC}
  
%   Uma referência~\cite{RealTimeRendering}
    
%    Uma referência~\cite{Wilhelms92}
    
%     Uma referência~\cite{BlinnAlbedo}
     
%      Uma referência~\cite{noise96}
       
%    Uma referência~\cite{IntervalTreeExplanation}
    
%    Uma nota de rodapé~\footnote{Nota .... Nota .... Nota .... Nota .... Nota .... 
%    Nota .... Nota .... Nota .... Nota .... Nota .... Nota .... Nota .... Nota .... Nota .... Nota .... Nota .... Nota .... Nota .... Nota .... Nota .... Nota .... Nota .... Nota .... Nota .... 
%    Nota .... Nota .... Nota .... Nota .... Nota .... Nota .... 
%    }
    
