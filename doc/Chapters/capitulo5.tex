\chapter{Implementação}
\label{cap:Implementacao}

% falta falar sobre filesystem vfs e como chegar ao socket file private\_data para alem da parte de interrupts, tasklets e bottomhalfs bem como sobre os contextos de execucao


\section{Ferramenta}
Considerações ...

\subsection{Repositório de dados}

De forma a manter a informação sobre quais os endereços e portos que uma
aplicação está a utilizar, sem que seja necessário consultar os sockets que
esta está a utilizar, foi necessário criar um repositório de dados onde
estivesse a informação relevante para as decisões do filtro de captura.

As abordagens para a criação de repositório de dados com base nas
estruturas de dados:

\subsubsection{Bitmap}

\subsubsection{Listas duplamente ligadas}

\subsubsection{Árvore}
Apesar de não ser necessário percorrer os elementos ou obter os indíces de
forma ordenada. 

\paragraph{Árvore Binária}


\paragraph{Arvores Balanceadas}
\textit{Red-black Tree} implementada no núcleo de operação permite efectuar

Complexidade de inserção, procura e remoção é O(log n), sendo n o número de
elementos da árvore.

Existe uma penalização no desempenho quando existe a necessidade de balancear a àrvore, mas os ganhos provenientes deste balanceamento face às àrvores não balanceadas é muito superior.

\subsubsection{Tabela de dispersão}
Não existe nenhuma implementação 

\subsubsection{Bloom Filter}
bloom filters ...

\paragraph*{}
A estrutura de dados escolhida para a criação do repositório de dados foi
\textit{Red-black tree}. Esta escolha foi devido principalmente à boa
performance e a existir uma implementação da estrutura de dados no núcleo do
linux. Devido a esta estrutura estar presente no núcleo do \textit{linux}
permitiu que a implementação da ferramenta tivesse um ciclo de implementação
mais rápido devido à confiança na ``validação`` desta estrutura pelos
responsáveis do núcleo do \textit{linux}.

\subsection{Funções a monitorizar}

Como foi anteriormente descrito em \ref{subsection:network} sobre as chamadas
aos sistema para utilizar as funcionalidades de rede

Foi necessário conhecer a \textit{ABI}\ref{ABI}\cite{ABI} referente a quais os
registos que são utilizados nas arquitecturas \textit{x86} e \textit{x86\_64},
para a passagem de parâmetros das chamadas aos sistema.

O sistema de \textit{KProbes} permite consultar os registos do processador onde
foi efectuada a chamada à função (neste caso a chamada ao sistema). Por isso
foram definidos 6 \textit{KRetProbes} referentes às chamadas \textit{connect},
\textit{bind}, \textit{accept}, \textit{sendto}, \textit{recvfrom} e
\textit{close}. Para esta última $\left( \textit{close} \right)$ devido à
elevada utilização desta chamada ao sistema por diferentes aplicações no
sistema, foi utilizada a função \textit{qqq coisa ...  ????}

\subsection{Filtro}

Como foi apresentado na arquitectura, na secção \ref{subsec:Filter}, o
\textit{overhead} acrescentado ao sistema mesmo sem a ferramenta estar activa é
a verificação da activação deste \textit{hook}.

\section{Informação dentro do núcleo}

Devido à necessidade de obter algumas informações sobre o estado da computação
dos diversos componentes da ferramenta foram criados alguns ficheiros no
sistema de ficheiros virtual \textit{DebugFs}.

Na verdade não foi só para obter o estado da computação foi também uma forma
invocar o inicio da monitorização. Isto é, através dos ficheiros \textit{pid},
\textit{ppid}, \textit{tgid} e \textit{option}. 

\subsection{Informação do processo}

Quando o módulo da ferramenta é executado os indicadores de \textit{pid},
\textit{ppid} e \textit{tgid} são inicializados com o valor -1.

Esta ferramenta de monitorização só irá passar a analisar os dados das chamadas
ao sistema quando estes valores estiverem definidos e identificarem o processo
que efectuou a chamada.

\subsubsection{Sistema de Ficheiros}

Dentro do sistema de ficheiros onde os \textit{sockets} estão incorporados,
como apenas estamos interessados em analisar os sockets construidos para a
familia \textit{AF\_INET} todos os sockets que não pertencerem a esta familia e
não forem do tipo \textit{TCP} ou \textit{UDP} não serão mais analisados. Os
\textit{sockets} que pertencem é então inicializado uma estrutura do tipo
\textit{não me lembro ??? ver código} onde estão definidos o porto, o endereço
e se são do tipo \textit{UDP} ou \textit{TCP}.

Dependendo da chamada ao sistema realizada entre a função definida na entrada
e a definida no retorno do \textit{KRetProbe}, são passados parâmetros que irão
ser utilizados na função de retorno.


\section{Ferramenta de testes}

Foi desenvolvida uma aplicação para testar o desenvolvimento da ferramenta.
Esta aplicação 

\subsection{Test Unit}

De forma a testar o repositório de dados foram efectuados diferentes
\textit{unit tests}, de forma a assegurar que todas as alterações efectuadas na
ferramenta ficavam de correctas.

\subsection{Aplicação de testes de performance}

Para ajudar na avaliação da ferramenta desenvolvida foi criado um conjunto de
aplicações independentes $\left( scripts e aplicações \right)$ para monitorizar
a actividade de algumas aplicações que se consideraram pertinentes no processo
de avaliação do desempenho da ferramenta.

A aplicação desenvolvida \textit{manager} tem de ser executada sob o controlo
do utilizador \textit{root}, devido à necessidade de executar processos que só
este utilizador tem acesso. Estes processos anteriormente mencionados são
\textit{insmod}, \textit{rmmod} e \textit{tcpdump}.

Os \textit{scripts} criandos e bash foram para poder obter o número de dados e
pacotes que foram transferidos na interface, bem como fazer a separação dos
tempos que as aplicações de testes correram para conseguir automatizar o
processo de execução e recolha de dados dos testes.

\subsection{DebugFs}

Criação de alguns ficheiros no sistema virtual \textit{configFs} para poder
observar o comportamento da monitorização e dos pacotes que chegaram ao sistema
de filtragem.

A forte restrição de apenas um valor por ficheiro do \textit{SysFs} obriga a
que para obter diferentes valores do sistema tenha de abrir diferentes
ficheiros. Como o sistema de ficheiros \textit{proc} se sitia principalmente
para os processos seria um bom candidado para a colocação de informação sobre 
