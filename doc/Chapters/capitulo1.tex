\chapter{Introdução}\label{cap:introducao}

\section{Contexto}
\label{sec:intro_context}
A monitorização de uma aplicação serve para se obterem informações relevantes acerca do seu comportamento.
Esta monitorização pode servir para verificar a correcção, recursos atribuídos e desempenho de execução de programas, etc.

A maioria dos sistemas de operação generalistas apresentam métodos de monitorização, devido à importância de que se revestem no desenvolvimento das aplicações.
Se algumas ferramentas são específicas na monitorização de determinados recursos (como a biblioteca \textit{LibPcap}, que é específica nas interacções com o exterior utilizando dispositivos de rede), outras são mais generalistas podendo monitorizar recursos diversos.

% A monitorização de rede de apenas um programa ainda não foi aplicada pelos principais fornecedores destes serviços, sendo uma operação complexa e muito depende dos sistema de operação em questão. 
O dinamismo das aplicações dificulta bastante o processo de monitorização.
Esta situação é particularmente sentida ao nível da monitorização dos dispositivos de \textit{IO}, nomeadamente nas interfaces de rede.
Nas aplicações onde o dinamismo das ligações é um dos seus pontos fortes, a monitorização é particularmente difícil de ser efectuada.

Apenas na camada de transporte da \textit{stack TCP/IP} se conhecem quais os portos de origem e destino de um determinado pacote.
Se pretendermos definir univocamente um fluxo de comunicação de rede para uma dada aplicação, é obrigatório conhecer o endereço origem e o porto de origem bem como o endereço de destino e o porto de destino.
Desta forma apenas quando o pacote chega à camada de transporte é que se consegue conhecer o seu destino.

Tendo em conta que aplicações \textit{P2P} utilizam diversos portos de comunicação, revela-se difícil capturar os pacotes com base nos actuais filtros, sem que se assista a uma elevada degradação do desempenho.
De modo que para capturar dados relevantes e simultaneamente minimizar os efeitos da monitorização, estes filtros são aplicados no interior do núcleo do sistema, logo que possível.

\section{Motivação}
\label{sub:intro_motivation}

No contexto de uma dissertação anterior~\cite{Farruca:2009}, utilizou-se a biblioteca \textit{LibPcap} para a captura do tráfego de rede, e um dos problemas resolvidos foi o de isolar os pacotes que pertencem a determinado processo alvo.
A solução conseguida não é fácil de integrar em qualquer outra ferramenta e oferece um fraco desempenho.
Esta dissertação centra-se nos mecanismos de inspecção oferecidos pelos sistemas de operação que permitem verificar o funcionamento e avaliar o desempenho do sistema e dos processos dos utilizadores (\textit{debugging}, \textit{profiling}, etc).

É objectivo desta dissertação investigar mecanismos, inclusive internos ao \textit{kernel}, que permitam incorporar a filtragem com base no identificador do processo e possível incorporação nas funcionalidades do \textit{pcap}, incluindo avaliações do \textit{overhead} introduzido e possíveis optimizações.

\section{Objectivos}
\label{sec:intro_goals}

----------------------------------------------------------------------------------------------------------

De forma a monitorizar uma aplicação utilizando o suporte do núcleo de sistema do \textit{linux}, com base no estudo prévio efectuado descrito no capitulo \ref{}.
A abordagem que se irá seguir utilizará um módulo dentro do núcleo de sistema de forma a instrumentar algumas funções da estrutura de rede, para obter os portos que um processo usa e quais os que irá utilizar num futuro próximo.
Irá fazer-se uso da infraestrutura de monitorização já presente no núcleo de sistema de operação do linux, como as apresentadas no capítulo \ref{}.
Os dados recolhidos irão ser utilizados na modificação do filtro utilizado no \textit{Linux Socket Filter} para capturar o tráfego.
De forma a permitir uma rápida adaptação do filtro às alterações no processo, a informação sobre os portos que estão a ser usados e suas alterações irão ficar monitorizadas e actualizadas.

Para que esta nova funcionalidade possa ser integrada na biblioteca \textit{LibPcap}, é necessário proceder a modificações ao nível da linguagem de mais alto nível necessária para se criar um filtro.


----------------------------------------------------------------------------------------------------------

\section{Principais contribuições} 
\label{sec:intro_contribuicoes}

A inclusão da possibilidade de monitorizar apenas o fluxo de rede de um determinado programa, poderá permitir que muitos dos filtros até agora construídos possam ser simplificados.
Para além desta simplificação a possibilidade de monitorização do fluxo de rede de uma aplicação, permite a observação dos dados sem ter a necessidade de um sistema prévio, que conheça os protocolos de mais alto nível aplicados sobre a rede.
Havendo a existência de um um sistema de captura de tráfego de rede genérico torna-se bastante benéfico para a análise de protocolos, pois nem sempre se tem acesso às especificações dos protocolos presentes nas aplicações.
Com este componente, a performance de obtenção dos dados relevantes poderá ser incrementada, mitigando anteriores problemas de performance entre o núcleo de sistema de operação e as ferramentas de análise de tráfego.





% Dificuldades em saber quais, de quem, ou porquê determinado pacote está dentro de uma rede pode ter desencadeado o inicio de diversos projectos, que têm como âmbito fazer uma análise dos pacotes que fluem numa rede.
% Se esta problemática desencadeou alguns projectos a verdade é que para a captura dos pacotes de um dado programa ainda não existe um projecto onde se insira a capacidade de análise dos pacotes, ou seja restringir o que se conhece como algo genérico como a captura de pacotes dentro de uma rede, para passar a ser a captura de pacotes de um dado programa. 



% \paragraph*{}
% Diversas vezes quando se procede ao debug de uma aplicação onde não existe o código fonte, é necessário fazer \textit{reverse enginnering}. Para se analisar os pacotes de dados que uma aplicação utiliza na comunicação através de rede é necessário capturar esses pacotes utilizando ferramentas como a biblioteca \textit{Pcap}. A \textit{LibPcap} consegue capturar os pacotes que transitam na rede mediante alguns filtros de forma a conseguir seleccionar os pacotes relevantes para a captura. A forma de filtrar os dados baseia-se nos filtros bpf\cite{Mccanne92thebsd} .
% 
% Uma vez que os filtros são baseados nos dados dos \textit{sockets}, é interessante e importante definir regras para os filtros que sejam baseados no identificador do processo de forma a poder obter todos os dados que são transferidos de forma transparente sem que seja necessário indicar todos os portos que a aplicação está a usar. Desta forma poder-se-á fazer um traço de execução de um processo não só baseado nas suas instruções como também baseado nos dados que são transferidos por uma aplicação. Utilizando esta abordagem podem ser mais facilmente analisados programas distribuidos que fazem uso das interfaces de rede para comunicarem.
\bigskip 





% Na secção \ref{sect:descricao} é apresentado são analisados os diversos problemas.
% Na secção \ref{sect:instrumentacao_casos_linux} são descritos os mecanismos de monitorização presentes no GNU/Linux de forma a poderem ser utilizados. 
% Na secção \ref{sect:LibPcap} é analisado o \textit{LibPcap} e a sua arquitectura de forma a conhecer toda esta biblioteca.
% Na secção \ref{sect:kernel_user_comm} são analisados diversos sistemas de transferência de dados de forma a poder obter informações do núcleo de sistema, e forma eficientes de transferir dados entre os dispositivos externos e o núcleo de sistema, e entre o núcleo de sistema e o espaço de utilizador.

\section{Organização do Documento}
\label{sec:intro_document_outline}


A continuação do documento está estruturado do seguinte modo:

\begin{itemize}
	\item \textbf{Capítulo \ref{cap:trabrelacionado} - \nameref{cap:trabrelacionado} - } Apresentação da temática monitorização, evidenciando a monitorização de rede. Apresentação do estado da arte da monitorização do núcleo do \textit{Linux} e trabalhos relacionados com a monitorização de processos

	\item \textbf{Capítulo \ref{cap:Estrutura} - \nameref{cap:Estrutura}  - } Neste capítulo é exposto a estrutura de comunicação e monitorização de rede do \textit{Linux} bem como os seus constituíntes. Apresentação da estrutura do \textit{MRoP} e da interligação com a estrutura de rede do \textit{Linux}.

	\item \textbf{Capítulo \ref{cap:Implementacao} - \nameref{cap:Implementacao} - } Implementação do \textit{MRoP} e discussão de alternativas de implementação

	\item \textbf{Capítulo \ref{cap:avaliacao} - \nameref{cap:avaliacao} - } Avaliação 

	\item \textbf{Capítulo \ref{cap:conclusao} - \nameref{cap:conclusao} - } 

\end{itemize}
