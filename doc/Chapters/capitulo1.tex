\chapter{Introdução}\label{cap:introducao}

\section{Contexto}
\label{sec:intro_context}
A monitorização de uma aplicação destina-se à obtenção de informações relevantes acerca do seu comportamento.
Não obstante, esta monitorização pode igualmente servir para verificar a correcção, recursos atribuídos e desempenho de execução, etc.

A maioria dos sistemas de operação generalistas apresentam métodos de monitorização, devido à importância de que estes se revestem, no desenvolvimento das aplicações.
Se algumas ferramentas são específicas na monitorização de determinados recursos (como a biblioteca \textit{LibPcap}, que é específica nas interacções com o exterior utilizando interfaces de rede), outras são mais generalistas podendo monitorizar recursos diversos.

O dinamismo das aplicações dificulta bastante o processo de monitorização.
Esta situação é particularmente sentida ao nível da monitorização dos dispositivos de \textit{IO}, nomeadamente nas interfaces de rede.
Nas aplicações onde o dinamismo das ligações é um dos seus pontos fortes, a monitorização é particularmente difícil de ser efectuada.

De modo a capturar dados relevantes, e simultaneamente minimizar os efeitos da monitorização, estes filtros são aplicados logo que possível, no núcleo do sistema.

Tendo em conta que aplicações \textit{P2P} utilizam diversos portos de comunicação, revela-se difícil capturar os pacotes com base nos actuais filtros, sem que se assista a uma elevada degradação do desempenho.
Não sendo exclusivo das aplicações \textit{P2P} a utilização de um elevado número de portos, esta elevada utilização de portos também é visível em sistemas de \textit{Voice over IP}.
Estas aplicações nas suas diversas comunicações não utilizam sempre portas conhecidas \textit{à priori}, pois por vezes fazem uso de protocolos em que na iniciação da sessão que negoceiam portos, o que dificulta bastante a utilização de filtros, pois é necessário conhecer todos os protocolos específicos das aplicações.
Para além desta situação existem aplicações que são executadas pelo administrador que não permitem a monitorização em nível utilizador, como é o caso da utilização do carregamento dinâmico de bibliotecas instrumentadas para efectuar a monitorização de aplicações.



\section{Motivação}
\label{sub:intro_motivation}

No contexto de uma dissertação anterior~\cite{Farruca:2009}, utilizou-se a biblioteca \textit{LibPcap} para a captura do tráfego de rede, sendo que o principal desafio solucionado foi isolar os pacotes que pertencem a determinado processo alvo.
A solução encontrada, não é de fácil integração em qualquer outra ferramenta, e oferece um fraco desempenho.

É objectivo desta dissertação investigar mecanismos, incluindo os internos ao núcleo, que permitam incorporar a filtragem com base no identificador do processo, a sua possível integração nas funcionalidades do \textit{pcap}, e proceder à avaliação da sobrecarga introduzida a respectiva optimização.

Esta dissertação centra-se nos mecanismos de inspecção, oferecidos pelos sistemas de operação que permitem verificar o funcionamento e avaliar o desempenho do sistema e dos processos dos utilizadores (\textit{debugging}, \textit{profiling}, etc).



\section{Principais contribuições} 
\label{sec:intro_contribuicoes}

A possibilidade de monitorizar apenas o fluxo de rede de um determinado programa, poderá permitir que muitos dos filtros construídos até ao momento possam ser simplificados.
Para além desta simplificação, a monitorização do fluxo de rede de uma aplicação, permite a observação dos dados sem necessitar de um sistema prévio, que identifique os protocolos de mais alto nível aplicados sobre a rede.
A existência de um sistema de captura de tráfego de rede genérico, torna-se benéfico para a análise de protocolos, pois nem sempre se tem acesso às especificações destes, presentes nas aplicações.
Com este componente, o desempenho na obtenção dos dados relevantes poderá ser incrementado, mitigando anteriores problemas de desempenho entre o núcleo de sistema de operação e as ferramentas de análise de tráfego.
Possibilidade de análise dos fluxos do processo sem a necessidade de instrumentar o código da aplicação, uma vez que a instrumentação é efectuada no núcleo.
Existe a possibilidade de monitorizar o fluxo de diferentes máquinas virtuais dentro de um sistema, desde que estas sejam implementadas utilizando processos, desta forma é possível individualizar e capturar o tráfego de cada uma.
Esta funcionalidade pode ser particularmente interessante em centro de dados, dado que não é possível efectuar esta análise ao tráfego sem a necessidade de parar a máquina.

\bigskip 


\section{Organização do Documento}
\label{sec:intro_document_outline}

Os restantes capítulos do documento, encontram-se assim distribuidos e estruturados:

\begin{itemize}
	\item \textbf{Capítulo \ref{cap:trabrelacionado} - \nameref{cap:trabrelacionado} - } Apresentação da temática monitorização, evidenciando a monitorização de rede. Apresentação do estado da arte da monitorização do núcleo do \textit{Linux} e trabalhos relacionados com a monitorização de processos

	\item \textbf{Capítulo \ref{cap:Estrutura} - \nameref{cap:Estrutura}  - } Neste capítulo é exposto a estrutura de comunicação e monitorização de rede do \textit{Linux} bem como os seus constituintes. Apresentação da estrutura do \textit{MRoP} e da interligação com a estrutura de rede do \textit{Linux}.

	\item \textbf{Capítulo \ref{cap:Implementacao} - \nameref{cap:Implementacao} - } Implementação do \textit{MRoP} e discussão de alternativas de implementação

	\item \textbf{Capítulo \ref{cap:avaliacao} - \nameref{cap:avaliacao} - } Avaliação funcional e de desempenho do \textit{MRoP} e dos seus componentes. Analisar o desempenho do sistema de monitorização utilizado (\textit{KProbes}.

	\item \textbf{Capítulo \ref{cap:conclusao} - \nameref{cap:conclusao} - } Apresentação das conclusões referentes à avaliação efectuada, apresentando algumas propostas de melhoramentos.

\end{itemize}
