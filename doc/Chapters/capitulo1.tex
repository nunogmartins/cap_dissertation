\chapter{Introdução}\label{cap:introducao}

% O texto da introdução, em fonte times-roman 12 e com um espaçamento e 
% meio,  deve apresentar (i) uma extensão ou introdução geral relativa ao resumo
% inicial, 
% (ii) uma contextualizando o trabalho, apresentando as suas motivações , (iii)
% uma 
% descrição clara do problema ou foco do trabalho e terminando com (iv) a
% aproximação 
% preconizada para a solução do problema ou do tratamento do tema focado, onde
% estejam 
% claras as contribuições previstas. Os alunos podem optar por apresentar esta
% introdução 
% endereçando os anteriores aspectos em sub-secções, como se exemplifica a seguir.



\section{Introdução geral} \label{sect:introducao}
% A introdução, escrita com fonte times-roman-12, pode ter, como referência
% indicativa, 
% entre 6 e 10 páginas, usando-se um espaçamento e meio.
% Dificuldades em saber quais, de quem, ou porquê determinado pacote está dentro de uma rede pode ter desencadeado o inicio de diversos projectos, que têm como âmbito fazer uma análise dos pacotes que fluem numa rede.
% Se esta problemática desencadeou alguns projectos a verdade é que para a captura dos pacotes de um dado programa ainda não existe um projecto onde se insira a capacidade de análise dos pacotes, ou seja restringir o que se conhece como algo genérico como a captura de pacotes dentro de uma rede, para passar a ser a captura de pacotes de um dado programa. 

A monitorização de uma aplicação serve para se obterem informações relevantes acerca do seu comportamento.
Esta monitorização pode servir para verificar a correcção, recursos atribuídos e desempenho de execução de programas, etc.

A maioria dos sistemas de operação generalistas apresentam métodos de monitorização, devido à importância de que se revestem no desenvolvimento das aplicações.
Se algumas ferramentas são específicas na monitorização de determinados recursos (como a biblioteca \textit{LibPcap}, que é específica nas interacções com o exterior utilizando dispositivos de rede), outras são mais generalistas podendo monitorizar recursos diversos.

% A monitorização de rede de apenas um programa ainda não foi aplicada pelos principais fornecedores destes serviços, sendo uma operação complexa e muito depende dos sistema de operação em questão. 
O dinamismo das aplicações dificulta bastante o processo de monitorização.
Esta situação é particularmente sentida ao nível da monitorização dos dispositivos de \textit{IO}, nomeadamente nas interfaces de rede.
Nas aplicações onde o dinamismo das ligações é um dos seus pontos fortes, a monitorização é particularmente difícil de ser efectuada.

Apenas na camada de transporte da \textit{stack TCP/IP} se conhecem quais os portos de origem e destino de um determinado pacote.
Se pretendermos definir univocamente um fluxo de comunicação de rede para uma dada aplicação, é obrigatório conhecer o endereço origem e o porto de origem bem como o endereço de destino e o porto de destino.
Desta forma apenas quando o pacote chega à camada de transporte é que se consegue conhecer o seu destino.

Tendo em conta que aplicações \textit{P2P} utilizam diversos portos de comunicação, é revela-se difícil capturar os pacotes com base nos actuais filtros, sem que se assista a uma elevada degradação do desempenho.
De modo que para capturar dados relevantes e simultaneamente minimizar os efeitos da monitorização, estes filtros são aplicados no interior do núcleo do sistema, logo que possível.

% \paragraph*{}
% Diversas vezes quando se procede ao debug de uma aplicação onde não existe o código fonte, é necessário fazer \textit{reverse enginnering}. Para se analisar os pacotes de dados que uma aplicação utiliza na comunicação através de rede é necessário capturar esses pacotes utilizando ferramentas como a biblioteca \textit{Pcap}. A \textit{LibPcap} consegue capturar os pacotes que transitam na rede mediante alguns filtros de forma a conseguir seleccionar os pacotes relevantes para a captura. A forma de filtrar os dados baseia-se nos filtros bpf\cite{Mccanne92thebsd} .
% 
% Uma vez que os filtros são baseados nos dados dos \textit{sockets}, é interessante e importante definir regras para os filtros que sejam baseados no identificador do processo de forma a poder obter todos os dados que são transferidos de forma transparente sem que seja necessário indicar todos os portos que a aplicação está a usar. Desta forma poder-se-á fazer um traço de execução de um processo não só baseado nas suas instruções como também baseado nos dados que são transferidos por uma aplicação. Utilizando esta abordagem podem ser mais facilmente analisados programas distribuidos que fazem uso das interfaces de rede para comunicarem.
\bigskip 

No \ref{cap:introducao} capitulo secção \ref{sect:descricao} é apresentada a temática de monitorização de aplicações.

Como a monitorização de rede é o principal objectivo desta dissertação, (subsecção \ref{sect:packet_capture}) apresenta-se de forma genérica a monitorização das interacções com o exterior (através de dispositivos de rede), evidênciando alguns dos principais desafios.

TODO
Sendo necessária a transferência de informação nos sistemas (subsecção \ref{sect:transf_information}) são descritos alguns dos principais factores.

No \ref{cap:trabrelacionado} capítulo são analisados os seguintes temas:
\begin{description}

\item 2.1 Monitorização no sistema de operação \textit{Linux}

\item 2.2 Transferência de dados dentro e fora do núcleo de sistema

\item 2.3 Biblioteca de monitorização de rede \textit{LibPcap}

\item 2.4 Outras abordagens de captura de pacotes.

\end{description}

Após ser analisado o \ref{cap:trabrelacionado} capítulo, propôs-se uma abordagem para a captura de pacotes de uma aplicação (secção \ref{sec:abordagem_proposta}).

% Na secção \ref{sect:descricao} é apresentado são analisados os diversos problemas.
% Na secção \ref{sect:instrumentacao_casos_linux} são descritos os mecanismos de monitorização presentes no GNU/Linux de forma a poderem ser utilizados. 
% Na secção \ref{sect:LibPcap} é analisado o \textit{LibPcap} e a sua arquitectura de forma a conhecer toda esta biblioteca.
% Na secção \ref{sect:kernel_user_comm} são analisados diversos sistemas de transferência de dados de forma a poder obter informações do núcleo de sistema, e forma eficientes de transferir dados entre os dispositivos externos e o núcleo de sistema, e entre o núcleo de sistema e o espaço de utilizador.

\section{Monitorização} \label{sect:descricao}

Pode ser necessário recorrer à monitorização para se compreender o comportamento das aplicações.
Esta permite-nos obter dados relevantes sobre os recursos utilizados pelas aplicações, de modo a proporcionar-nos um conhecimento mais profundo do seu comportamento.
As informações recolhidas podem servir para analisarmos comparativamente o desempenho de diferentes versões de uma mesma aplicação, porquanto ao conseguir-se obter dados da utilização do \textit{cpu}, da memória, dos dispositivos de \textit{IO}, etc, é possível compreender o comportamento dinâmico das aplicações, daí que é comum ser utilizada como auxiliar na depuração de programas.~\cite{DuartePhd05}.

% Fazer uma captura dos pacotes de um dado programa pode parecer simples, mas na verdade não o é, principalmente devido ao dinamismo que certos programas apresentam. Uma forma de obter informações sobre o comportamento dos programas é através de algumas estruturas presentes no núcleo de sistema. Desta forma uma das vertentes que irá ser analisada nas secções seguintes é a instrumentação de código dentro do núcleo de sistema.

\subsection{Obtenção de informação}\label{sect:instrumentation_overview}
% 
% %Falta falar sobre o que é a instrumentação
% Instrumentação de uma aplicação é uma técnica de obtenção de informações sobre o estado e a performance desta. Assim permite também realizar \textit{debugging} ao código.
% 
% %Esta parte é sobre a parte de ser dinâmico e as vantagens que isso traz 
% Ao ser dinâmica é muito importante, pois permite que sejam efectuadas análises sem que o núcleo do sistema tenha de ser recompilado e posteriormente reiniciado.Desta forma pode ser utilizado em sistemas em produção.
% 
% Alguns sistemas de análise de performance utilizam este tipo de técnicas para poder obter, dados precisos sobre o estado da computação.
% 
% Quando os programas são criados os compiladores geram estruturas para poder ajudar no processamento e estruturação do código. (DWARF)
% 
% 
% A instrumentação dinâmica permite que ao ser alterado o código, mesmo que este esteja a correr sobre a instanciação de um \textit{debugger} este seja efectuado, esta instrumentação continue a ser efectuada.
% 
% \paragraph*{O que é}

%\paragraph*{Para que serve}
Os dados relativos ao comportamento das aplicações, obtidos de diferentes fontes, são coligidos e posteriormente analisados por ferramentas especializadas.
Existem diferentes formas de coligir, visualizar e até interactuar com as ferramentas de monitorização, cada uma com as suas especificidades e capacidades próprias.

Tendo em vista o conhecimento e um melhor aproveitamento das capacidades de cada uma, são apresentadas as diferentes formas de análise / avaliação:

\begin{itemize}
 \item 


Se tivermos como referência a visualização da história dos eventos, podemos analisá-los de duas formas distintas:
\subparagraph*{Online - }
Enquanto decorre a monitorização da aplicação é possível ir observar os dados que são recolhidos pela aplicação.
Como os eventos estão a ser recolhidos e visualizados em simultâneo, apenas podemos observar a história até ao momento.

\subparagraph*{Offline ou \textit{Post-Mortem} - }
A história do programa é analisada após este se ter completado (razão pela qual é utilizada a designação \textit{Post-Mortem}).
Este método permite-nos analisar toda a sua história.

\item

Se a opção de monitorização se basear na interactividade do utilizador, é possível defini-la de duas formas:

%PODER MODIFICAR DURANTE A MONITORIZACAOO pode reagir a modificar a aplicacao
\subparagraph*{Activa - }

Por iniciativa explícita do utilizador é possível inquirir o sistema de monitorização sobre o estado da computação.
Esta é a forma com maior interactividade, uma vez que permite ir analisando e modificando os parâmetros da monitorização.
Este método é muitas das vezes descrito como \textit{computacional steering}.

%Modificar os pontos de monitorização ou apenas analisar os dados ao mesmo tempo que estão a ser capturados. Desta forma o utilizador pode controlar explicitamente a monitorização da aplicação. Este controlo pode ser apenas visualização da computação ou pode ser sobre a forma como se está a desenvolver a monitorização podendo ser baseado num ciclo de monitorização, decisão sobre os pontos a monitorizar (adicionar, remover, etc).

\subparagraph*{Passiva - }
Esta forma de monitorização é utilizada principalmente em ambientes onde é mais importante a obtenção de todos os dados e apenas no final nos debruçarmos sobre a sua análise.
Esta forma designada de passiva, por o utilizador não tem intervenção sobre a forma como os dados estão a ser adquiridos, o que reduz a perturbação do sistema. 
%Quando não se tem interesse em ir alterando os pontos a serem monitorizados, está é uma das escolhas correctas pois permite um menor grau de perturbação do sistema.
\item

A instrumentação pode ser de dois tipos: a estática e a dinâmica, cada uma com as suas características próprias.

\subparagraph*{Estática - }

Na instrumentação estática o código instrumentado é definido em tempo de compilação, utilizando bibliotecas próprias para o efeito tais como a utilização da função \textit{assert()}, que define os pontos a serem monitorizados.
Durante a monitorização não podem ser adicionados ou removidos pontos de análise.

\subparagraph*{Dinâmica - }

Em contraste com a instrumentação estática está a dinâmica, é mais complexa e permite a inserção e remoção dos pontos a serem monitorizados.
O dinamismo verifica-se pela ausência do ciclo $introduzir ponto\rightarrow compilar programa\rightarrow executar\rightarrow remover ponto$.
A utilização de pontos de instrumentação dinâmica pode ajudar a reduzir o grau de perturbação, uma vez que apenas são definidos os pontos que se desejam observar.
\end{itemize}

\subparagraph*{}
Todavida é no uso combinado da história dos eventos, da interactividade e da instrumentação, que reside a melhor forma de monitorização.

\subparagraph*{
%Recolha de dados
}
A recolha de informação é uma das componentes mais sensíveis relativamente ao grau de perturbação da monitorização.
Em geral os dados vão sendo armazenados num \textit{buffer} em memória.
Caso exista algum evento que assinale que o \textit{buffer} está cheio ou uma indicação explícita que se pretende armazenar esses dados, para posterior análise, estes são transferidos para memória persistente.

\subparagraph*{
%Grau de perturbação
}
Conforme o anteriormente mencionado, existe a preocupação de que o sistema a ser monitorizado tenha um baixo grau de perturbação.
Daí que diversas abordagens foram criadas para reduzirem o impacto da monitorização num sistema em produção.
Uma destas abordagens foi a utilização de instruções especializadas, que alguns processadores têm para \textit{debug}, de modo a utilizar os recursos que mais se adequam à monitorização.
Nem sempre são utilizados estes métodos, pois estão demasiados dependentes da arquitectura, o que dificulta a sua portabilidade.
Com vista a minimizar a perturbação do sistema alguns sistemas de monitorização utilizam uma técnica de amostragem, que permite obter indicações sobre os estado da computação ao fim de um certo número de iterações.
Esta técnica, em oposição à criação de um traço de execução, permite obter dados sobre os recursos apenas por amostra, enquanto que na criação de um traço de execução é possível obter todos os dados de forma a criar uma história completa.

\paragraph*{
%Apresentação dos dados recolhidos
}
No entanto, capturar estes dados oriundos da monitorização pode revelar-se insuficiente, se não dispusermos de uma ferramenta onde estes possam ser trabalhados, de modo a obter análises mais ricas visualmente.

\subsection{Monitorização de Rede}\label{sect:packet_capture}

% \paragraph*{}
% Os dados que circulam nas redes de computadores, em geral utilizam um conjunto de protocolos de forma a poderem comunicar através de diferentes dispositivos. Para saber que dados chegam a um determinado dispositivo, via interface de rede, é necessário conseguir obter uma cópia destes. Esta cópia pode ser efectuada logo nos dispositivos de rede, ou quando o pacote de dados chega à máquina de destino, e aí é o núcleo de sistema da maquina de destino que irá se encarregar de efectuar a cópia e entregar o pacote à aplicação/camada de destino.
% 
% 
% 
% \subparagraph*{Aumento de performance da captura em redes de alta velocidade}
% \paragraph*{}
% OBSERVAR AS APLICACAES NO CONTEXTO DA REDE 
% \paragraph*{}
% Diferentes abordagens têm sido estudadas de modo a aumentar a velocidade de captura dos pacotes que chegam a uma dada interface de rede.
% Estas abordagens centram-se mais na forma de diminuir o número de cópias que são efectuadas (entre os diferentes \textit{buffers}) e o local onde os dados irão estar disponíveis. Estas duas situações permitem um aumento da velocidade de captura dos pacotes.
% 
% Algumas destas abordagens são: \textit{MMAP}, \textit{Zero Copy}, \textit{Ring Buffers}, \textit{NAPI - New API}, estas abordagens são apresentadas na secção Transferência de dados no núcleo de sistema (\ref{sect:kernel_user_comm})
% 
% \subparagraph*{O problemas da perda de pacotes}
% Quando o núcleo de operação está a capturar os pacotes que atravessam a interface de rede, caso não consiga acompanhar o ritmo de chegada de pacotes à interface poderá perder pacotes. Estas perdas de pacotes, não são aceitaveis uma vez que se os pacotes forem perdidos, mais tem de ser retransmitidos, o que irá desencadear o efeito bola de neve sobre o número de pacotes que a rede tem de pedir para serem retransmitidos. Desta forma a velocidade degrada-se ainda mais, as filas de pacotes aumentam e poderá degenerar num \textit{DOS (Denial of Service)}.Estas perdas são contabilizadas de forma a poderem ser analisadas, de forma a remediar a situação de perda de pacotes em futuras operações.
% 
% ATRASAR 
% OVERHEAD 
% 
% COMPORTAMENTO DA APLICACAO

Em geral as ferramentas de monitorização da interacção do processo com o exterior, como seja via rede, são baseadas na captura de pacotes de forma passiva.
As ferramentas capturam os pacotes que fluem na rede para posterior análise ao tráfego, que pode insidir sobre a largura de banda utilizada, principais protocolos, eventuais problemas de segurança, etc.

\subparagraph*{Dinamismo das aplicações}
Como foi descrito nas subsecções \ref{sect:instrumentation_overview}, as aplicações são dinâmicas e a este dinamismo se devem algumas dificuldades com que nos deparamos ao monitorizar as aplicações.
Relativamente à monitorização das interacções com o exterior, utilizando a interface de rede, este dinamismo é de novo um factor chave, pois existem dificuldades na forma de identificar os fluxos pertencentes a um processo.

\subparagraph*{Formas de reduzir o volume de dados utilizando filtros}
A utilização de filtros  na captura do tráfego que circula na rede é uma forma eficiente de obter apenas os dados que se mostrem relevantes com vista à satisfação nos nossos objectivos e são particularmente importantes quando o volume de dados que circula na rede é extremamente elevado.
Tendo em vista a eficiência, estes filtros são implementados no núcleo do sistema de operação, baseando-se em regras simples, que podem ser combinadas para contemplar situações mais complexas.

\subparagraph*{Dificuldade de criação de filtros dinâmicos}
Os filtros actualmente suportados para capturar pacotes no \textit{Linux}, são definidos \textit{a priori}, e não há forma eficiente de os alterar dinamicamente, visto que a captura tem de ser parada a fim de ser criado um novo filtro, posteriormente aplicado, sendo depois retomada a captura dos pacotes, de acordo com as novas regras.

\subparagraph*{Filtros mais complexos e inteligentes}
De forma a aumentar a performance da captura de pacotes, os filtros são aplicados o mais cedo possível, ou seja logo quando chegam da interface de rede.
Face a esta situação, o tipo de filtros que se podem aplicar têm de ser escritos numa linguagem simples.
Quando os filtros são demasiado complexos, os módulos do núcleo têm de passar grande parte da informação para as camadas superiores de forma a poderem ser aplicados filtros mais elaborados, que conseguem um nível mais abstracto sobre a informação que presente no \textit{payload} dos pacotes.

\subparagraph*{Problemas com o aumento do overhead}
Como foi apresentado na subsecção \ref{sect:instrumentation_overview} a perturbação da aplicação monitorizada devido ao \textit{overhead} da captura e execuções resultantes desta, podem provocar diferentes problemas.

\subparagraph*{Outros problemas
% Comportamento da aplicação, atrasos, perda de pacotes, entre outros
}
Se devido à monitorização de rede, os pacotes que chegam ao programa de monitorização necessitam de um tratamento pesado, este tratamento pode levar a que o sistema não consiga responder de forma correcta às aplicações que fazem uso das comunicações externas, pois o sistema pode não conseguir emitir \textit{acknowledges} dos pacotes que anteriormente tenham chegado, ou então não conseguir tratar novos pedidos de pacotes uma vez que os \textit{buffers} ficam cheios e começam a descartar os anteriores pacotes, aumentando ainda mais os problemas de retransmissão e de atrasos.

\subparagraph*{Técnicas para aumento da performance}
% \todo{performance ou desempenho}
Diferentes técnicas têm vindo a ser desenvolvidas para aumentar a performance de monitorização das interfaces de rede.
Como foi visto anteriormente o problema do aumento do \textit{overhead} é muito importante e por isso deve ser mantido bastante reduzido, ao reduzir o \textit{overhead} de certa forma também se está a aumentar a performance de captura.

Uma nova \textit{API} de atendimento de interrupções foi criada com este propósito.
Esta nova \textit{API} está analisada na subsecção \ref{par:NAPI}.

\subparagraph*{Utilização de processadores \textit{multi-core} na captura}
Numa época em que a utilização de máquinas equipadas com processadores multi-core é uma realidade para a grande maioria do público, será uma boa aproximação utilizar as capacidades dos novos processadores de forma a ultrapassar algumas dificuldades sentidas na captura de pacotes.
Este é já um tópico que já está a ser estudado por algumas pessoas ligadas ao desenvolvimento da arquitectura de rede.


\subsection{Transferência de informação}\label{sect:transf_information}
%problemas de transferencia no overhead, o que aumenta mto as trocas de contexto e a frequencia de transferencia

% Como existe uma arquitectura estruturada dentro do sistema de operação para comunicar, esta deve ser respeitada. Por questões de eficiência diversas vezes esta estrutura é desrespeitada em detrimento do aumento das transferências de dados.

Transferir informação de um ponto para outro dentro de um sistema quase sempre envolve uma cópia de dados.
Esta cópia de dados pode se demasiado pesada na performance do sistema, especialmente se o meio for de transmissão lenta e o número de \textit{bytes} envolvidos nesta transferência for grande, ou se a frequência com que esta é efectuada for bastante grande.
Estas transferências/cópias de dados para além da cópia em si necessitam de trocas de contexto entre o núcleo de sistema e as aplicações que necessitam dos dados, pois apenas o núcleo de sistema pode efectuar instruções privilegiadas de forma a garantir a segurança e robustez do sistema.
Diferentes técnicas têm sido desenvolvidas de forma a aumentar a performance destas operações. 


\section{Descrição do problema} \label{sect:descricao_prob}
 No contexto de uma dissertação anterior~\cite{Farruca:2009}, utilizou-se a biblioteca \textit{LibPcap} para a captura do tráfego de rede, e um dos problemas resolvidos foi o de isolar os pacotes que pertencem a determinado processo alvo.
A solução conseguida não é fácil de integrar em qualquer outra ferramenta e oferece um fraco desempenho.
Esta dissertação centra-se nos mecanismos de inspecção oferecidos pelos sistemas de operação que permitem verificar o funcionamento e avaliar o desempenho do sistema e dos processos dos utilizadores (\textit{debugging}, \textit{profiling}, etc).

É objectivo desta dissertação investigar mecanismos, inclusive internos ao \textit{kernel}, que permitam incorporar a filtragem com base no identificador do processo e possível incorporação nas funcionalidades do \textit{pcap}, incluindo avaliações do \textit{overhead} introduzido e possíveis optimizações.

\section{Abordagem proposta}\label{sec:abordagem_proposta}
De forma a monitorizar uma aplicação utilizando o suporte do núcleo de sistema do \textit{linux}, com base no estudo prévio efectuado descrito no capitulo \ref{}.
A abordagem que se irá seguir utilizará um módulo dentro do núcleo de sistema de forma a instrumentar algumas funções da estrutura de rede, para obter os portos que um processo usa e quais os que irá utilizar num futuro próximo.
Irá fazer-se uso da infraestrutura de monitorização já presente no núcleo de sistema de operação do linux, como as apresentadas no capítulo \ref{}.
Os dados recolhidos irão ser utilizados na modificação do filtro utilizado no \textit{Linux Socket Filter} para capturar o tráfego.
De forma a permitir uma rápida adaptação do filtro às alterações no processo, a informação sobre os portos que estão a ser usados e suas alterações irão ficar monitorizadas e actualizadas.

Para que esta nova funcionalidade possa ser integrada na biblioteca \textit{LibPcap}, é necessário proceder a modificações ao nível da linguagem de mais alto nível necessária para se criar um filtro.

\section{Principais contribuições previstas} \label{sect:contribuicoes}

%  As principais contribuições previstas devem poder ser descritas em não mais do
% que uma página, podendo adoptar-se, 
% por exemplo, um estilo de apresentação por itens, com uma pequena descrição de
% um 
% parágrafo associado a cada item.

A cada necessidade nasce uma oportunidade.
Para resolver a necessidade de determinar quais os pacotes de rede que uma determinada aplicação manipula, nasceu a oportunidade de ajudar a comunidade com uma nova ferramenta.

A inclusão da possibilidade de monitorizar apenas o fluxo de rede de um determinado programa na \textit{LibPcap}, poderá permitir que muitos dos filtros até agora construídos possam ser simplificados.
Para além desta simplificação a possibilidade de monitorização do fluxo de rede de uma aplicação, permite a observação dos dados sem ter a necessidade de um sistema prévio, que conheça os protocolos de mais alto nível aplicados sobre a rede.
Havendo a existência de um um sistema de captura de tráfego de rede genérico torna-se bastante benéfico para a análise de protocolos, pois nem sempre se tem acesso às especificações dos protocolos presentes nas aplicações.
Com este incremento à biblioteca \textit{LibPcap}, a performance de obtenção dos dados relevantes poderá ser incrementada, mitigando anteriores problemas de performance entre o núcleo de sistema de operação e as ferramentas de análise de tráfego.

% A inclusão de uma nova instrução no filtro do linux (\textit{Linux Socket Filtering}) e a consequente aplicação na biblioteca \textit{LibPcap}, de forma a pode ser aplicada em conjunto com os anteriores filtros. Assim esta nova instrução pode ser aplicada de forma transparente para os utilizadores da \textit{LibPcap}. Esta inclusão poderá permitir que o número de regras dos filtros possam diminuir uma vez que em geral podem ser definidos várias regras para obter os pacotes de uma determinada aplicação, com esta inclusão esta situação poderá ser reduzida ou mesmo eliminada.
